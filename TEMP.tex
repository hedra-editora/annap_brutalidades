\section{O náufrago}\label{o-nuxe1ufrago}

Ele trepava como um náufrago. Eu me agarrava a ele tentando não afundar.
``Eu te amo Anna'', ele dizia com a boca no meu ouvido e o pau na minha
buceta, ``Eu te amo, minha linda'', e metia mais forte, ``Te amo
muito'', e mais forte, ``Muito'', mais forte, ``Muito!'' Eu gritava de
prazer e desespero. ``Também te amo, meu amor. Eu só quero você, só
você, só você!'' O suor do rosto dele pingava em meus olhos e chorava a
lágrima mais salgada. ``Eu também só quero você, Anna!'' Eu bebia o suor
do seu pescoço e nos beijávamos sufocando meus gemidos. \emph{Ainda
vamos nos afogar, ainda vamos nos afogar...} Só tínhamos para beber a
saliva um do outro naquele quarto pobre de hotel. Ele gritava ``Eu te
amo, meu amor!'' e seu rosto sofria enquanto seu pau me estocava com
toda força. Eu perdia as palavras e respondia aos gritos. Eu me meu
náufrago morremos toda semana nos braços um do outro. Ele diz ter por
mim um amor maior que tudo; eu tenho casa, marido, família e filhos
planejados. E agora choro o suor dele, me afogo nos beijos dele, quero
morrer nos braços dele. Eu, que já beijei os olhos da morte, essa morte
que me é tão familiar, eu, que tinha planos, tinha uma família, agora
vejo tudo rodar à minha volta enquanto me afogo. Tudo vem à tona, mas
submirjo. Vejo através da água a minha vida na superfície: a casa
reformada e mobiliada com carinho, a cortina recém comprada ainda por
pendurar, o abajur para a mesa da sala. Vejo a luz do abajur através da
água. Vejo o rosto magoado do meu marido. Mas me afogo e afundo no suor
do meu náufrago até tudo escurecer. Ele me beija. ``Você é a mulher da
minha vida, Anna!'' A paixão me aprisiona. Vejo o juízo girar nas águas
da superfície. Dentro d'água é escuro e eu sufoco. Meu marido me beija e
eu sufoco. ``Fica comigo, Anna. Eu construo uma vida com você!'' Uma
vida submersa. Ele se agarra a mim como um náufrago, eu me agarro a ele
e submirjo.

\section{Mar}\label{mar}

O poeta\\
Contempla o mar\\
E depois canta\\
Eu vou lá\\
Mergulho inteira\\
E depois conto

\section{Investimento}\label{investimento}

\textbf{P.} 13 de julho de 2011 às 11:58

Saudade docê -- acho que viciei

\textbf{B.} 13 de julho de 2011 às 11:59

Eu também tô com saudade amor

\textbf{P.} 13 de julho de 2011 às 12:13

A mancha de sangue não saiu do lençol....

\textbf{B.} 13 de julho de 2011 às 12:29

Nunca mais vai sair

\textbf{P.} 13 de julho de 2011 às 12:30

Rs

\textbf{B.} 13 de julho de 2011 às 13:15

Tô quase terminando o Kafka. Se ninguém me brochar eu termino hoje ainda

\textbf{P.} 13 de julho de 2011 às 13:16

Tá gostando?

\textbf{B.} 13 de julho de 2011 às 13:17

Estou criando uma admiração pelo Gregor

Que no meu entender vai se foder do começo ao fim

Mas ele é foda

Muito louco

Um pouco alienado

Mas quem não é?

Sofremos como o Gregor o tempo todo

\section{No canto do estoque}\label{no-canto-do-estoque}

\textbf{B.} 14 de julho de 2011 às 19:49

Pô não aguento mais

Quero ir embora

Tô ficando louco aqui

\textbf{P.} 14 de julho de 2011 às 19:49

Tá com pouco trabalho?

\textbf{B.} 14 de julho de 2011 às 19:50

Trampei pra caralho até às seis

Agora tá foda

Tô sem nada pra fazer

Não dá pra ler porque tiram minha concentração

Tô aqui escondido no canto do estoque

\section{Ternura freestyle}\label{ternura-freestyle}

Puta

Canalha

Cachorro

Cadela

Patife

Piranha

Gostoso

Gostosa

Vagabundo

Vadia

Moleque

Biscate

Filho da puta

Gostosa

Gostoso

Moleca

Bandido

Sem vergonha

Ordinário

Safada

Indecente

Vulgar

Sórdido

Cachorra...

\section{Paixão}\label{paixuxe3o}

Aquele papelzinho que vai para a máquina no bolso da calça sujando toda
roupa que você pensava ter lavado.

\section{Desconfiança}\label{desconfianuxe7a}

\textbf{P.} 15 de novembro de 2011 às 12:08

Tô angustiada

\textbf{B.} 15 de novembro de 2011 às 12:08

Por quê?

\textbf{P.} 15 de novembro de 2011 às 12:08

Tô muito triste com a nossa discussão

\textbf{B.} 15 de novembro de 2011 às 12:09

Anna eu te amo

Tô contigo. Agora, se você se sente sozinha mesmo assim, eu não sei bem
o que fazer

\textbf{P.} 15 de novembro de 2011 às 12:10

Eu queria que você parasse de desconfiar de mim. Entenda: eu não gosto e
não sei trair

\textbf{B.} 15 de novembro de 2011 às 12:10

Já parei

\textbf{P.} 15 de novembro de 2011 às 12:11

Mas é chato ficar pisando em ovos

\textbf{B.} 15 de novembro de 2011 às 12:13

Concordo Anna

Vou cuidar de mim

\textbf{P.} 15 de novembro de 2011 às 12:13

Eu \textsc{preciso} que você faça isso

Não deixa a neurose vencer o amor

\textbf{B.} 15 de novembro de 2011 às 12:16

Vou me policiar

Vou depositar minha confiança em você

Aí é contigo

Faz o que for melhor

Você é livre

Eu não sou você

\textbf{P.} 15 de novembro de 2011 às 12:20

Sou livre, por isso sou mais responsável que a maioria

É isso que eu quero que você entenda

\textbf{B.} 15 de novembro de 2011 às 12:25

Então

Fica em suas mãos meu amor

Minha confiança é sua

\textbf{P.} 15 de novembro de 2011 às 12:26

E nas suas mãos também, porque eu também confio em você

\textbf{B.} 15 de novembro de 2011 às 12:34

Vou lutar por você

E ver o que a vida me reserva

É o que me resta

Estou cansado demais

Só estou com muito medo de me machucar

\textbf{P.} 15 de novembro de 2011 às 12:36

Seu ciúme me joga num lugar horrível

Ciúme não é amor

\section{Soneto submerso}\label{soneto-submerso}

Meu náufrago tem olhos frágeis

Doídos, cílios compridos

Um procura algum sentido

O outro sente, impotente

Meu náufrago me submerge

E quero levá-lo à tona

Mas usa pedras nos pés

E vive na subzona

Respiro água faz tempo

Ando no fundo do mar

Com as narinas tapadas

E os olhos marejados

Porque não basta o amor

No abraço dos afogados

\section{Vagabunda}\label{vagabunda}

\textbf{B.} 23 de dezembro de 2011 às 11:07

Desculpa

\textbf{P.} 23 de dezembro de 2011 às 11:07

Quero desculpar

Mas quem desculpa é o tempo e as mudanças de atitude

\textbf{B.} 23 de dezembro de 2011 às 11:41

Eu só quero seu bem

\textbf{P.} 23 de dezembro de 2011 às 11:41

Mas está me fazendo mal

\textbf{B.} 23 de dezembro de 2011 às 11:42

Eu não pensei mal de você, eu só disse o que pensava, mas sem querer te
magoar

Só não quero que você seja uma mulher malvista. Não por você, mas pelos
outros

\textbf{P.} 23 de dezembro de 2011 às 11:43

Eu sempre vou ser mal vista, eu sou mulher

\textbf{B.} 23 de dezembro de 2011 às 11:46

Então meu amor, aí que tá. Seja mais escrota com quem é realmente
escroto contigo

Você tá entendendo?

\textbf{P.} 23 de dezembro de 2011 às 11:48

Eu entendo, só não concordo

Tô muito cansada. Vou deitar um pouco, depois a gente conversa

\textbf{B.} 23 de dezembro de 2011 às 18:53

Tá tudo bem contigo?

\textbf{P.} 23 de dezembro de 2011 às 18:53

Não. Só durmo e choro

\textbf{B.} 23 de dezembro de 2011 às 18:54

Fala então

\textbf{P.} 23 de dezembro de 2011 às 18:54

Não me saem da cabeça as palavras que você
\protect\hypertarget{vagabunda-1}{}{}usou

Tem palavras que a gente não deve usar nunca

\section{Puta}\label{puta}

Ninguém mais leva a sério

O seu vitupério

\section{Não vai dar certo}\label{nuxe3o-vai-dar-certo}

Mas pode ser

que agora

eu prefira o caminho errado

ao correto

\section{``Ela gosta é dos
coitadinhos''}\label{ela-gosta-uxe9-dos-coitadinhos}

\protect\hypertarget{foda}{}{}Gosto de homens que trepam para hoje,
porque amanhã ninguém sabe o que será. Difícil encontrar isso na classe
média. A classe média tem futuro. Só encontrei entre pessoas que vieram
das periferias e refugiadas. Um dia os amigos do Beltrano disseram para
ele:

\section{Lúmpem}\label{luxfampem}

Lúmpem, margem de manobra, massa acrítica, herança escravista,
inorgânico, proletariado excedente, trabalhador informal, inimpregável,
pobre coitado, nem nem, subproletariado, sobrepopulação trabalhadora
superempobrecida permanente, não qualificado para o mercado de trabalho,
classe em si, despolitizado, precarizado, da horda dos ressentidos, para
lá da ponte, vagabundo nato, moldado para o encarceramento, ou
extermínio, sobrevivente, maconheiro, zé ninguém, fulano, Beltrano.

\section{Os intensos se atraem}\label{os-intensos-se-atraem}

E se destroem

\section{Dupla jornada}\label{dupla-jornada}

(Junho de 2013)

Congelei. É isso. Congelei. Não acredito. Não posso me mexer. No
entanto, algo se move dentro da minha estagnação. Não é possível. Não é
possível que qualquer coisa se produza dentro de tamanha paralisia. Mas
está acontecendo. De que forma isso se justifica? Estou sozinha nisso?
Alguém vai me ajudar? De forma alguma. Estou sozinha. Ninguém. Nada.
Pessoas continuam nascendo na catástrofe. Coisas também. A vida
melhorou. E ainda assim é insuportável. Minha mãe nasceu no nazismo. Eu
nasci na ditadura. Queria ter gerado bem antes. Mas o que se criaria nos
últimos vinte anos? Agora essa democracia. Tolerância ao intolerável.
Falta de criatividade. Sobrevivência anestésica. Felicidade é não doer.
Da minha janela vejo o beco. Beco não, cortiço. Tudo me dói. Tudo me
dói. Não consigo me mover. Ninguém vai me ajudar? Ninguém vai a lugar
nenhum. Ninguém se desloca. Nem se o outro estiver morrendo. Não doer é
minha maior ambição. Foda-se a felicidade. Preciso me alimentar. E não
me movo. Será que morri enquanto sobrevivia? Estou criando ou morrendo?
Estou reproduzindo porque não me matei? Alguém vai chegar aqui? Há
alguém entre mim e o cortiço? Tenho fome. E se eu morro? Nada nasce? Não
tenho certeza se quero não poder morrer mais. E se a vida for inviável?
O que vai ser do que engendrei? Estou paralisada. Como faço para comer?
Preciso me levantar. Preciso me mover. Estou enlouquecendo. Ouço vozes.
Não tem ninguém. Ouço vozes. Vejo coisas. Vejo os ratos deixando os
cortiços e se tornando pessoas. Ou ao menos parecendo. Vejo pessoas se
movendo. Estão na rua. Uma só voz. Creio que sonho, mas é bonito e me
alimenta. Já posso me sentar. Não estou na rua, mas me movo. Vozes e
pessoas em movimento. Algo se desloca. Agora posso conceber. Mesmo sem
saber o que se produz. Ou reproduz. Mesmo que seja um devaneio. Algo já
pode nascer. Inclusive nada. Mas agora é possível.

\section{Esgotamento}\label{esgotamento}

\textbf{P.} 22 de janeiro de 2014 às 13:51

Eu tô doente de stress e você não tá levando a sério. O Quim tá sofrendo
porque a mãe e o pai não deram conta. Eu tô tendo crises de choro e de
ansiedade direto e tô fazendo um esforço enorme pra arrumar essa bagunça
toda. Então, por favor, não me ligue em tom de cobrança.

\textbf{B.} 22 de janeiro de 2014 às 13:51

Anna

Tô te cobrando nada não

Tô preocupado com vocês

\textbf{P.} 22 de janeiro de 2014 às 13:52

Eu não tô te ligando porque sai caro

A gente se fala por mensagem ou por aqui

\textbf{B.} 22 de janeiro de 2014 às 13:54

Tô com saudade dele

\textbf{P.} 22 de janeiro de 2014 às 13:55

Eu também tô triste que vocês estão longe

Mas teve que ser assim

\textbf{B}. 22 de janeiro de 2014 às 13:56

Vai ficar até quando?

\textbf{P.} 22 de janeiro de 2014 às 13:56

Não sei, B.

Até ficarmos bem

\textbf{B.} 22 de janeiro de 2014 às 14:01

O Quim tá mamando bastante

Fico preocupado com o tanto de mamada que ele dá por dia

Isso tá te sugando

\textbf{P.} 22 de janeiro de 2014 às 14:04

Mas a falta de sono suga muito mais

\section{Positive Vibration}\label{positive-vibration}

Tive de descobrir na prática que não é a potência da droga que determina
o grau de dependência -- e sim a falta de perspectivas. E quando essa
ausência vem revestida de um estiloso rastafári, boa música e infinitos
rituais coletivos de Jah, fica muito difícil encarar como vício. Daí
para a frente, qualquer fresta de possibilidade será desfibrada,
enrolada, tragada e asfixiada pela dependência. E cumpre-se o círculo
redundante.

Toda metade masculina de Rio das Voltas vivia anestesiada assim. Jovens
deprimidos em festa. Deprimidos na festa do sol.

\section{Recuperação}\label{recuperauxe7uxe3o}

\textbf{B.} 24 de janeiro de 2014 às 12:58

Tá tudo bem entre a gente Anna?

\textbf{P.} 24 de janeiro de 2014 às 13:17

B., vou falar meio interrompido por causa do bebê

Eu tô chateada de ter chegado a esse ponto de cansaço físico e mental

De ter vindo pedir penico pra minha mãe

E sobretudo de ver que nosso filho não estava bem

Acho que falhamos, eu e você

Como casal e como pais

E estou tentando me recuperar disso também

\textbf{B.} 24 de janeiro de 2014 às 13:19

Anna

Não tem falha alguma

Ele vai ficar bem

\textbf{P.} 24 de janeiro de 2014 às 13:20

Vai ficar, mas não estava

E eu também não estou

E você também não

\textbf{B.} 24 de janeiro de 2014 às 13:22

Anna quem tá doente é você

O neném tá sentindo o reflexo disso

E agora não é hora pra neurose de ter jogado a toalha

Força, mulher

\section{Aquela grana}\label{aquela-grana}

\textbf{B.} 24 de janeiro de 2014 às 13:28

Anna você sabe quando volta?

\textbf{P.} 24 de janeiro de 2014 às 13:33

Ainda não tenho ideia

\textbf{B.} 24 de janeiro de 2014 às 13:34

Então meio chato falar

Mas aquela grana deve durar só até esse fim de semana

Tudo bem se você mandar algo?

\section{Manga}\label{manga}

Sob a fina flor da pele

A carne boa no gosto

Não tinha nem cabimento

Era fruto do desgosto

Parecia não ter fim

Ao cabo de toda carne

De cabo a rabo caroço

Deu cabo de todo gosto

Acabou dizendo o azedo

Pareceu mangar de mim

\section{Vinte quilos}\label{vinte-quilos}

\textbf{B.} 27 de janeiro de 2014 às 10:21

Como você tá?

\textbf{P.} 27 de janeiro de 2014 às 10:50

Ainda no osso

E perdendo um quilo por semana

Já foram 15

\textbf{B.} 27 de janeiro de 2014 às 10:52

Mas por qual motivo?

Saúde?

Ou ele tá mamando muito?

\textbf{P.} 27 de janeiro de 2014 às 10:52

Acho que os dois

Porque o cansaço me tira o apetite

Então tô comendo por obrigação

E meu estômago encolheu muito

O Quim tá começando a mamar menos

Essa noite já dormiu um pouco mais entre as mamadas

\protect\hypertarget{mais-cinquenta}{}{}Ele já tá feliz de novo

\section{Orides}\label{orides}

O amor

Capenga de quatro pernas

E única mão

Esse amor

Não

\section{Retomada}\label{retomada}

\textbf{B.} 29 de janeiro de 2014 às 21:48

Me atende

Anna só atende por favor

Eu não tô legal

Olha eu não sei o que tá rolando

Mas eu só queria ouvir sua voz

Ia me fazer bem agora

Eu tô numa puta crise de ansiedade

Só precisava falar contigo um pouco

Só pra me sentir mais seguro

\textbf{P.} 30 de janeiro de 2014 às 07:15

Você enlouqueceu?

\textbf{B.} 30 de janeiro de 2014 às 08:36

Anna

Tive uma noite horrível ontem

Mas tô tentando melhorar

Me desculpa

Errei

Foi desespero

Só tive uma noite ruim

\textbf{P.} 30 de janeiro de 2014 às 08:57

Eu tô tendo noites horríveis desde que o Quim nasceu e pouquíssima
solidariedade sua

Porque no fundo você acha que a responsabilidade de cuidar da cria é só
minha

\textbf{B.} 30 de janeiro de 2014 às 09:04

Que isso?

De onde você tirou isso?

\textbf{P.} 30 de janeiro de 2014 às 09:04

Você não consegue entender quando eu digo que não aguento mais

\textbf{B.} 30 de janeiro de 2014 às 09:04

Nunca foi assim

Você não tá com a cabeça legal

Eu também não

Mas a gente tá nessa junto

\textbf{P.} 30 de janeiro de 2014 às 09:05

Você não toma iniciativa de procurar soluções, pediatra, livros, tentar
fazer ele dormir sem precisar de mim

Dormir mais cedo pra acordar melhor pra cuidar dele

Você acha que eu sempre posso aguentar mais e se responsabiliza muito
pouco

\textbf{B.} 30 de janeiro de 2014 às 09:06

Tenho te falado coisas boas

Dando demonstração de afeto e carinho

Te apoiando aí

\textbf{P.} 30 de janeiro de 2014 às 09:06

Só que não era pra eu ter precisado vir

\textbf{B.} 30 de janeiro de 2014 às 09:07

Não era mesmo

Mas você foi

\textbf{P.} 30 de janeiro de 2014 às 09:07

Porque você não me deu suporte suficiente!

\textbf{B.} 30 de janeiro de 2014 às 09:07

O queee?

\textbf{P.} 30 de janeiro de 2014 às 09:07

Segurar o bebê pra eu poder mijar e tomar banho não é suficiente

\textbf{B.} 30 de janeiro de 2014 às 09:07

Para de falar comigo como se eu fosse um filho da puta

Não faz isso

Eu tô dando minha vida pra você

\textbf{P.} 30 de janeiro de 2014 às 09:08

Eu corri sozinha atrás de pediatra, diarista, alimentação, lembrar
vacinas, fazer a bolsa dele pra sair, tudo

Fui eu que mudei minha vida toda pra você poder se erguer

\textbf{B.} 30 de janeiro de 2014 às 09:09

Para de falar desse jeito

Você vai chegar onde eu não quero

\section{Sentido de
autopreservação}\label{sentido-de-autopreservauxe7uxe3o}

Eu não tinha nenhum

\section{Rompimento}\label{rompimento}

\textbf{P.} 31 de janeiro de 2014 às 07:10

B., Meu amor

Eu sinto sua falta o tempo todo

Falta dos momentos em que a gente conseguia rir juntos, cada vez mais
raros, mas tão bons!

Falta do seu olhar de amor pra mim

Do seu cheiro quando eu te abraçava na cama

Do sexo tão intenso que só a gente faz

Eu deito todo dia naquela cama e lembro do amor maravilhoso que a gente
fez quando eu me separei

Seu sorriso de amor e carinho olhando pra mim

A entrega dos nossos corpos

A alegria de estar juntos

Acho que posso dizer que você foi a maior paixão da minha vida

Por você e pela gente eu fiz coisas que não fui capaz de fazer por mais
ninguém

(Tenho que atender o bebê, depois continuo)

\textbf{P.} 31 de janeiro de 2014 às 08:53

Nós dois, em função desse grande amor, enfrentamos família e amigos a
ponto de perder nossas referências no mundo. Você saiu do lugar que você
ocupava, bom ou ruim, pra fundar um novo lugar comigo. E eu também
deixei um mundo no qual eu, mal ou bem, me localizava, pra construir
outra vida com você.

Não, querido, minha situação não tem nada de confortável. E no meio
desse sofrimento enorme eu ainda tenho que me virar do avesso pra não
deixar nosso filho sofrer

Tá difícil, dá vontade de morrer só pra poder descansar um pouquinho

(Deixa eu tentar deitar o Quim)

\textbf{B.} 31 de janeiro de 2014 às 09:19

Você vai continuar comigo?

\textbf{P.} 31 de janeiro de 2014 às 09:36

Amor, deixa eu dizer tudo

Nossa história de amor é incrível do ponto de vista de tudo o que
enfrentamos pra ficar juntos. E enfrentamos com muita coragem.

Mas há a parte em que falhamos, falhamos miseravelmente. Falhamos um com
o outro, como casal, como homem e mulher. E o mais doído é que essa
parte talvez nem tenha a ver com as dificuldades externas que tivemos
que enfrentar.

\textbf{B.} 31 de janeiro de 2014 às 09:39

Atende aí

\textbf{P.} 31 de janeiro de 2014 às 09:40

Não posso atender que o Quim tá dormindo aqui

Deixa eu falar, depois falamos por telefone

\textbf{P.} 31 de janeiro de 2014 às 09:44

Vou continuar escrevendo porque esse é o tempo que eu tenho

Daqui a pouco ele acorda

Eu preciso falar da parte em que falhamos

A gente não pode esquecer que mesmo com tanto amor, estamos pra terminar
desde que começamos

Primeiro era a minha separação

Depois as inseguranças de falta de grana e o ciúme

Nossa convivência era feita de um afeto imenso, mas de muitas brigas
também

Me entristecia tanto computador e tão pouca atividade juntos em casa, me
entristecia a gente precisar de maconha pra rir juntos de forma um pouco
mais descontraída. E, vamos admitir, talvez a maconha estivesse lá pra
compensar uma grande falta de afinidade.

Falhamos muito na convivência, eu me ressentindo do seu mau-humor, da
falta de empenho em deixar o ambiente mais leve dentro de casa; você se
ressentindo da minha incapacidade de me desvincular de quem me faz mal.
Nessa história ninguém é inocente

E a toda hora, em todos esses três anos, a vida a dois foi difícil, não
vamos mentir. E a gente nunca deixou de falar em separação, como nunca
deixamos de desejar construir uma vida juntos.

Até que, no final de 2012, depois de dois anos juntos e insuficientes,
estávamos esgotados dessa luta, estávamos desistindo. E fomos viajar e
foi ainda pior. Não só porque entrou o meu lixo familiar na história, o
que não foi pouco, mas porque nós já não estávamos conseguindo ficar
bem. Era cerveja de dia, vodca até a embriaguez à noite. Eu, magoada, já
não conseguia me entregar na cama. E você se ressentia disso. E tinha
certeza que nós íamos terminar depois dessa viagem

Só que finalmente engravidei. E então renovamos nossa fé na marra mesmo
e partimos pra mais uma jornada que, infelizmente, foi um pesadelo

Dessa parte não vou conseguir falar muito porque dói demais ainda. E
quase morremos os dois, eu e Quim, de abandono, de tristeza, mas
sobrevivemos

Quando ele nasceu, minha vida estava arruinada -- está ainda. Eu não
tinha mais doutorado, nem emprego, nem forças pra voltar. Estávamos
sozinhos com uma criança, perdidos, sem saber o que fazer

A decisão de ir pra Rio das Voltas não foi exatamente um desejo, nem meu
nem seu. Lá tinha a esperança de algum apoio familiar, mas sobretudo eu
tinha a certeza de que aí era o melhor lugar pra você começar sua vida
profissional

Eu também acreditei que poderia me inserir em alguma faculdade, mas acho
que nisso errei. Pelo visto a cidade aí só funciona na base do favor de
algum ``político'' e isso nunca vai me beneficiar

(Preciso empurrar alguma comida, depois continuo)

\textbf{B.} 31 de janeiro de 2014 às 10:41

Anna fala comigo

Só preciso saber que decisão você tomou diante disso tudo

Não preciso passar por isso

Eu tô mal aqui

\textbf{P.} 31 de janeiro de 2014 às 11:10

B., tô tentando falar as coisas direito, com delicadeza, dentro da
complexidade delas. E tô pensando enquanto escrevo.

Mas como não sei quando vou poder continuar a escrever, vou falar o que
você quer.

Eu não quero voltar pra vida que a gente tava levando aí.

Também não quero ficar aqui, mas não tô vendo outro jeito agora.

Nossa vida tava escura, cheia de brigas e ataques desde o processo de
mudança.

E a gente teve fé e foi em frente mesmo com nossos atritos avisando que
era melhor parar e pensar.

Mas a vida aí, numa casa escura, cheia de brigas, em uma cidade estranha
e sem conhecer ninguém, e com um recém nascido... eu não dou conta

Você pode acreditar que é covardia minha, tudo bem.

Mas lembra que quando não era eu que ameaçava ir embora, era você. Que
muitas vezes você também disse que não queria mais. E você não queria
mesmo, porque a vida tava ruim demais e ninguém tem que querer isso.

A minha tortura psicológica, desde sempre, é a ideia de afastar o filho
do pai e da irmã. B., eu fui até onde eu podia, mas quando ele começou a
ficar mal, a ficar inseguro e sem mãe, então não deu mais pra vacilar.

Eu tô destruída com tudo isso e ainda por cima não deixei de te amar.

\textbf{B.} 31 de janeiro de 2014 às 11:20

Anna, você tá se separando de mim mesmo?

Sério?

É isso mesmo que tá acontecendo?

\textbf{P.} 31 de janeiro de 2014 às 11:21

Estou, assim como você também quase foi embora várias vezes, agora eu
estou indo. E é terrível tudo isso, mas é isso mesmo que está
acontecendo.

\section{Tanta gente me chama}\label{tanta-gente-me-chama}

de musa

de deusa

de fusa

Mas não me ama

\section{Eu só queria o seu bem, meu
bem}\label{eu-suxf3-queria-o-seu-bem-meu-bem}

\textbf{B.} 31 de janeiro de 2014 às 11:21

Anna só atende a última vez o telefone aí

Eu vou aí pra ver o Quim

E resolver isso contigo direito

Jamais pensei que você ia fazer essa molecagem comigo

\textbf{P.} 31 de janeiro de 2014 às 11:49

Tudo bem B., vem sim

Eu não quero terminar por telefone ou mensagem

\textbf{B.} 31 de janeiro de 2014 às 11:50

Na boa, mudei de ideia, não vou praí não. Atende o telefone aí

\textbf{P.} 31 de janeiro de 2014 às 11:50

B. não dá, tô com o bebê no colo

Vamos nos falar de noite

\textbf{B.} 31 de janeiro de 2014 às 11:52

Não

\textbf{P.} 31 de janeiro de 2014 às 11:52

Ele já tá nervoso

\textbf{B.} 31 de janeiro de 2014 às 11:53

Quando você tiver uma resposta clara sobre isso a gente conversa. Eu
quero você, te amo mas tô decepcionado contigo. Você me abandonou

\textbf{P.} 31 de janeiro de 2014 às 11:55

Preciso cuidar da nossa cria. Desculpe, vou desligar porque a prioridade
é ele. À noite falamos.

\textbf{B.} 31 de janeiro de 2014 às 11:56

\protect\hypertarget{o-tuxe9dio-o-uxf3dio-e-o-nojo}{}{}Anna cuida dele
aí 11:56 mas 11:56 eu tô totalmente magoado contigo 11:57 você me faz de
bobo 11:57 jamais faria isso contigo 11:57 vim pra cá pra Rio das Voltas
com você 11:57 cheio de esperança 11:57 determinação 11:57 mesmo na
tristeza que a gente tá passando 11:57 O Quim precisa da gente 11:57
mais você tem a certeza que aí 11:58 longe de mim 11:58 vai ser melhor
pra ele 11:58 você poderia estar aqui 11:58 com os cuidados da minha tia
aqui em casa 11:58 com quatro mil na conta pra contratar qualquer pessoa
pra te ajudar com ele 11:58 e eu te ajudando do jeito que dá 12:00 eu tô
aqui num vazio danado 12:00 me sentindo um idiota 12:00 por te
acreditado em você 12:00 que você era forte 12:00 que mesmo na escuridão
você não ia me abandonar 12:01 sou um babaca mesmo 12:01 eu confiando em
você 12:01 acreditando que essa porra vai passar 12:01 que a gente vai
vencer mais uma 12:01 e você me volta pros seus inimigos 12:02 você vai
criar o Quim sozinha 12:02 quem vai te ajudar de verdade? 12:02 e vai
deixar ele com quem? 12:02 na creche? 12:02 com sua mãe? 12:02 você não
querer ficar comigo eu entendo 12:03 mas você tá fodendo a vida dele
pensando dessa forma tão emocional 12:03 ele precisa estar com quem
gosta dele 12:03 ele gosta de mim 12:03 você tá me tirando o meu filho
de perto de mim 12:04 e eu tô rezando aqui pra não surgir ódio se isso
acontecer 12:04 eu sei que ele não tava bem 12:04 eu falhei em alguns
aspectos 12:04 você também 12:04 ai você foge do problema 12:04 vai
fazer eu acreditar que eu me casei com uma mulher fraca covarde 12:09 e
você com essa sua cabeça cheia de merda 12:09 tá achando que vai
resolver a porra toda aí 12:09 na casa da sua mãe 12:10 vai viver a sua
vida de mulher adulta 12:11 você tá acabando com o sentimento mais puro
que eu tenho 12:11 o amor que sinto por você 12:15 em nenhum momento
arreguei contigo 12:15 pensei em terminar sim 12:15 mas eu terminei?
12:15 terminei? 12:15 te deixei na mão? 12:16 te abandonei? 13:06 tô
muito mal 13:30 deixa pra lá 13:30 não liga mais pro que eu falei 13:31
tô com medo e confuso 13:31 inseguro e magoado 13:31 mas entenda 13:31
só queria seu bem

\section{Felicidade é não doer}\label{felicidade-uxe9-nuxe3o-doer}

O mundo me dói o tempo todo. Café, álcool, espírito natalino,
oportunidades, tudo me descompensa. Detesto tomar remédio, mas vivo em
eterna hipocondria. É raro quando acordo e nada dói.

\section{Uma fonte inesgotável}\label{uma-fonte-inesgotuxe1vel}

\textbf{P.} Sábado, 1 de fevereiro de 2014 às 06:42

B., tô arrasada por causa de ontem

Quim ficou nervoso, inseguro, apavorado mesmo

Quase não dormiu

Eu perdi todo o trabalho que eu tava fazendo com tanta dificuldade pra
ele dormir

Novamente tô levantando da cama sem energia pra cuidar dele

E ele tá bem mal

Então, pensa no seu filho e tenta compreender

Eu vou desligar aqui

\textbf{B.} 1 de fevereiro de 2014 às 06:45

Não

\textbf{P.} Sábado, 1 de fevereiro de 2014 às 06:45

Não vou atender o telefone

\textbf{B.} 1 de fevereiro de 2014 às 06:46

Não Anna para com isso

para de me rejeitar

\textbf{P.} Sábado, 1 de fevereiro de 2014 às 06:46

Não vou falar com você hoje porque tenho, \textsc{tenho} que cuidar do
nosso bebê

\textbf{Beltrano}, 1 de fevereiro de 2014 às 06:47 \textsc{anna você me
quer como um inimigo mesmo eu só tô querendo ficar bem por que você não
colabora eu não tô com a mínima vontade de fazer mal pra vocês então
para de me tratar como um monstro você é melhor que isso de onde você
tirou essa ideia de que eu tô te fodendo que seu tivesse cuidado de você
direito não precisava você ta aí como anna eu me dedicando pra você aqui
vendo você ficar doente do meu lado você tá me ferrando se realmente uma
é prioridade por que você me ataca dessa forma e depois foge não me
encara de frente tá começando a me deixar com raiva de você e acho que é
isso mesmo que você quer que eu fique com raiva ódio fale merda perca a
linha que aí fica fácil pra você correr do barco ou você tá de muita mas
muita maldade não conheço gente ruim de verdade você tá em outra classe
já porque você tem que me matar porque você não tem a coragem dignidade
postura de mulher de me encarar de frente correndo de mim cara eu não
sou seu inimigo você diz que tá mal que tá sozinha com a cria aí cadê a
sua mãe cadê os milhões de amigos aí sei acho que quem tá na situação de
ter comidinha prontinha só chegar e comer tem um babá pra cuidar de você
e do quim e ainda uma fonte inesgotável de grana você tá falando mais de
você o tempo todo do que de mim mesmo então esse era o ditado que você
costumava falar alguma coisas de pedro sei lá}

\section{Tá doendo}\label{tuxe1-doendo}

--- Tudo bem?

--- Tudo indo.

\protect\hypertarget{o-que-jouxe3o-diz-de-pedro-fala-mais-sobre-}{}{\protect\hypertarget{o-que-jouxe3o-diz-de-pedro-fala-mais-sobre--1}{}{}}

\section{Vergonha}\label{vergonha}

\textsc{só sei na boa eu posso tá morrendo de desespero aqui mas eu tô
limpo sei que fiz o melhor e você é você foda-se né você não sente culpa
mesmo se acha no direito de me encanar tirar meu filho de perto de mim
dizer coisas desnecessárias sabe qual é seu problema você precisa tá
sempre num lugar onde as pessoas falem o que você quer escutar tomem
atitudes sempre baseadas em você você não sabe lidar com a culpa sempre
passa a peteca pra frente você se engana o tempo todo inventa coisas pra
sua cabeça cria seu mundo e se acha especial na boa para caralho durante
essa semanas que eu tô sozinho aqui pude observar melhor as coisas cara
quanta mulher pobre comendo pão com manteiga pra sobreviver e minha
esposa cheia das oportunidades fazendo isso é muito triste ver que o
pessoal que trabalha na coleta tem varias mulheres sem dente fudida
esses dias uma senhora me pediu água toda sem graça pô a gente trocando
ideia aí começamos a falar da vida um cadim coisa de dois minutos e ela
falando que já ouviu o Quim gritando aí eu falei que Quim é foda grita
mesmo dando risada e a mulher me vira e fala nossa tenho cinco lá e eu
nossa quanto filho pô a mulher levanta sete da manhã dá conta de geral
sai pro batente que é ficar limpando guia volta pra casa e termina o
rock e na moral ela tá sozinha não tem marido não tem mãe anna diante
daquilo eu senti vergonha você entende vergonha }

\section{Pai}\label{pai}

Para mim era uma espécie de pedra no sapato, um elemento opressivo que
minha mãe dizia que eu tinha que educar.

Durante minha infância, repetia periodicamente: Você não é bonita, você
é charmosa. Seu irmão é que é bonito.

Depois de adulta, o assunto era sobre pais que estupravam filhas.

Hoje ele insiste em histórias de idosos que foram abandonados para
envelhecer e morrer sozinhos.

\section{}\label{section}

\section{Lirismo contemporâneo}\label{lirismo-contemporuxe2neo}

Doutor, a angústia voltou.

Dobramos o Lyrica?

\section{Voltei para a minha classe com o rabo entre as
pernas}\label{voltei-para-a-minha-classe-com-o-rabo-entre-as-pernas}

Voltei com o rabo entre as pernas e uma criança de cinco meses no colo.

Voltei para o bairro onde eu nasci e voltei para a universidade.

Voltei para a família e para as amigas de antes.

Em nada me reconheço.

\section{Sobre o luto}\label{sobre-o-luto}

Não tive tempo

Ele só tinha seis meses

\section{Sobre o amor}\label{sobre-o-amor}

--- Mãe, como foi que você casou com o meu pai?

--- Ele foi a única pessoa que teve coragem de me esconder na casa dele
na época da ditadura.

--- Pai, como foi que você casou com a minha mãe?

---
\protect\hypertarget{orides-1}{}{\protect\hypertarget{sobre-o-amor-ii}{}{}}Um
dia ela foi jantar na minha casa e nunca mais saiu.

\section{O que não nos mata nos
fortalece}\label{o-que-nuxe3o-nos-mata-nos-fortalece}

Mas eu preferia ser mais bailarina e menos halterofilista

\section{Rivotril}\label{rivotril}

É horrível

Mas sou sensível

\section{24/7}\label{section-1}

Preciso pagar o café

Preciso pegar o livro

Preciso passar na farmácia

Preciso estudar para a prova

Preciso fazer render

Preciso fazer feijão

Preciso terminar hoje

Amanhã é feriado

Preciso deixar a casa arrumada

Ele chega muito cansado

Preciso estar bem quando ele chegar da creche

Preciso melhorar logo dessa dor

Preciso dormir para poder completar meus sonhos

\section{O amor como sonhar pelo
outro}\label{o-amor-como-sonhar-pelo-outro}

``Tive um sonho tão bonito com você! Era um fim de tarde de um dia
quente. Eu fui para uma estreia de um projeto de dança seu. Era quase um
solo. Você estava gravidíssima, quase nove meses! E o Vladmir Herzog era
codiretor da peça. Tinha um curta-metragem interagindo com você em cena.
Um curta-metragem em preto e branco contando o dia do casamento de um
casal na neve e você dançando tão, tão poderosa e ao mesmo tempo tão
sozinha.

Em uma parte, o palco foi tomado por homens muito diferentes em tipo de
corpo e altura, mas todos com a mesma roupa, mesmo cabelo, mesma barba e
chapéu. E eles ocuparam o palco de forma homogênea, como uma plantação
de eucaliptos que se move, mas se move muito lentamente. Então o vídeo
congela, essas pessoas todas em cena com você, eles quase imóveis e você
fluindo no meio deles com uma certeza e uma agilidade linda, enquanto a
voz do Herzog cantava uma música de ninar.

Acordei chorando de tão bonito que foi.''

\section{Macia}\label{macia}

--- Quer deitar na minha almofada, mamãe? Ela é macia...

\section{Verdade}\label{verdade}

--- Você vai pra escola e mamãe vai trabalhar.

--- De quê?

--- Mamãe é escritora.

--- Não é não.

--- E qual o trabalho da mamãe? Cuidar do Quim?

--- É, cuidar do Quim.

--- Verdade. Mas quando você vai pra escola, o trabalho da mamãe é ler e
escrever.

--- Aaahh, isso não parece muito bom...

--- Por que, filho?

--- Porque é muito chato!

\section{Você consegue muitas
coisas}\label{vocuxea-consegue-muitas-coisas}

--- Ih, filho, não tô conseguindo fazer esse fantoche de meia, tô
ficando chateada.

--- Ah, você consegue muitas coisas, não fica chateada!

\section{Febre}\label{febre}

--- Quim tá bebê.

--- Tá bebê, filho?

--- Tá.

--- Tudo bem, filho. Quando a gente fica doente fica meio bebezinho
mesmo.

-- É? Num sabio...

\section{Crescer}\label{crescer}

--- Precisa comer pra crescer, não é mamãe?

--- É, filho, pra crescer, pra ficar forte...

--- Pra carregar coisas pesadas... Quando você ficar pequenininha, eu
carrego pra você!

\section{Entendimento}\label{entendimento}

--- Ih, filho, começou a chover! Bem na hora que a gente ia pra praia!
Que droga, não vai dar pra ir.

--- É, que droga, não é?

--- É, que droga mesmo. Mas vamos aproveitar pra almoçar que eu tô
morrendo de fome.

--- Que bom que choveu, não é mamãe?

--- Por quê?

--- Porque você tava com fome.

\section{Adormecer}\label{adormecer}

--- Eu vou ficar olhando pra você, tá mamãe? Porque você é muito
bonitinha.

\section{Dia dos pais}\label{dia-dos-pais}

\section{Quim já parou de acordar antes das seis. Eu ainda
não.}\label{quim-juxe1-parou-de-acordar-antes-das-seis.-eu-ainda-nuxe3o.}

\section{Dorme, mamãe}\label{dorme-mamuxe3e}

--- A mamãe...

--- Filhinho...

--- A mamãe...

--- Sim, filho, eu sou sua mamãe.

--- E eu sou o papai?

--- Não, seu papai chama B. Você tem um papai, mas ele tá longe.

--- É? Eu tenho um papai?

--- Tem, seu papai é o B.

--- E ele tá longe?

--- Ele tá longe, filho. Você que ver fotos dele?

--- Quero!

--- Olha, seu papai é esse aqui, o B. Ele é bonito...

--- (Rindo) É engraçado!

--- (Chorando) É engraçado?

--- É engraçado.

--- Esse é o seu papai.

--- Tá bom, mamãe.

--- (Chorando) Tá bom?

--- Tá.

--- (Chorando)

--- Dorme, mamãe.

\section{Beijinho}\label{beijinho}

--- Olha, ele não parece muito bem.

--- Quem não parece muito bem?

--- O Quim.

--- Quer um beijinho pra sarar?

--- Não, não adianta.

\section{Abraçamento}\label{abrauxe7amento}

--- Hoje quando você voltar da escola vai ter arroz e feijão fresquinho,
franguinho com batatas, tomatinho e salada de alface.

--- E você vai fazer tudo isso sozinha, sem ajuda?

--- É, filho.

--- Mas se você precisar, você me chama que eu venho ajudar você, tá
mamãe?

\section{Meu bem querer}\label{meu-bem-querer}

--- Quim, você vai brincar na casa do vizinho mas eu não vou ficar, tá?

--- Tá, mamãe, descansa um pouquinho enquanto a gente brinca.

\section{\texorpdfstring{Entendimento
\textsc{ii}}{Entendimento ii}}\label{entendimento-ii}

--- Mas como eu saí da sua barriga se não foi pela pepeca?

--- A médica cortou a minha barriga e tirou você lá de dentro.

--- E depois?

--- Depois ela costurou tudo bem direitinho e ficou tudo bem.

--- Que bom, mamãe, fico feliz que você tenha ficado bem.

\section{- De nunca!}\label{de-nunca}

--- Quim, para de bagunçar as coisas!

--- É porque eu tô bravo!

--- Mas não adianta você ficar brava comigo!

--- É porque... Eu não gosto de você!

--- Até parece... De onde você tirou essa ideia?

\section{Amor/humor}\label{amorhumor}

--- Você é meu filho preferida.

--- Mas eu sou o único!

--- Ainda bem. Senão o outro corria o risco de perceber a minha
preferência enorme por você.

\section{Alta ficção}\label{alta-ficuxe7uxe3o}

Li para ele uma história onde a cabra briga com o Lobo a noite inteira.

--- Mas ela ficou toda machucada de verdade ou de brincadeira?

--- De verdade na história, mas a história é uma brincadeira.

--- Há! Brincadeira! Eu sabia disso!

\section{Odisseu}\label{odisseu}

--- Boa noite, amormeuzinho.

--- Eu não sou melzinho!

--- Não, você é meu amorzinho.

--- Eu não sou seu amorzinho!

--- Não? Então quem é o meu amorzinho?

--- Ninguém.

--- Tá bom. Boa noite, Ninguém.

\section{Pestanas}\label{pestanas}

Imersa na contemplação da criança enquanto aguardo que ele adormeça...

--- Mãe, de tanto você não piscar o olho você vai ficar sem pestana!

\section{Do amoi}\label{do-amoi}

Pela manhã, como de costume, Quim me cobre de abraços e beijos:

--- Ai, filhinho, eu gosto tanto de você! Sabe o que eu mais gosto de
você?

--- Eu sei! Do amoi!

\textbf{Sobre a força}

- Ai, mamãe, você me machucou!

- Desculpa! Eu puxei com muita força, né? Às vezes eu esqueço que sou
forte...

- É, você esquece todo dia!

\section{\texorpdfstring{Do amoi
\textsc{ii}}{Do amoi ii}}\label{do-amoi-ii}

--- Me abraça forte? Mais forte que uma pisada de alossauro!

\section{No meio da noite}\label{no-meio-da-noite}

--- Mamãe! Acende a luz! Me dá colo! Você tava escurecida...

\section{O futuro é uma câmera}\label{o-futuro-uxe9-uma-cuxe2mera}

--- Uma, o que é futuro?

--- O futuro é uma câmera.

--- Uma câmera?

--- É, por onde a gente vê as coisas!

\section{O futuro a quem pertence?}\label{o-futuro-a-quem-pertence}

Imagino Beltrano construindo uma vida para si, podendo se tornar pai de
Quim, meu amigo. E deixando todo sofrimento no passado. Quim, seu pai
não está mais longe.

Ou imagino ele morrendo de bala, garrafada, overdose, suicídio. E
deixando todo sofrimento no passado. Quim, seu pai morreu.

\section{Desistência}\label{desistuxeancia}

Hoje pedi para Quim escolher um brinquedo para doar. Ela pegou o boneco
que parece o Beltrano e deu.

\section{Procuro um homem em quem
confiar}\label{procuro-um-homem-em-quem-confiar}

Vou escrever no aplicativo

Não

Não vou

\section{Logo depois ele morria}\label{logo-depois-ele-morria}

Sonhei que começava uma história com um homem de olhos bons e corpo
aconchegante.

\section{Queria um homem que me
amasse}\label{queria-um-homem-que-me-amasse}

Mas não fosse louco por mim

\protect\hypertarget{sexo}{}{\protect\hypertarget{sexo-1}{}{}}

\textbf{O real e o simbólico}

--- Mamãe, finge que eu era um cachorrinho e eu não tinha papai, o meu
papai morreu quando eu nasci. E finge que eu tava perdido e não sabia
onde tava a minha mamãe...

\section{Que fazer?}\label{que-fazer}

Clima de funeral no país

Na cidade

Na universidade

Ninguém sabe

\textbf{Faz tempo que não leio literatura}

Não estou suportando a realidade

\section{Sexo}\label{sexo}

Não faço mais.

\section{Meu sonho erótico}\label{meu-sonho-eruxf3tico}

Estava falando com um amigo sobre um intelectual que a gente admirava
muito e comentava, sem muita importância, que ele havia feito uma
generalização boba ao dizer que mulheres não gostam de água gelada.

Meu amigo tentava ponderar dizendo que, em geral, mulheres sentem mais
frio que homens etc. Eu o olhava divertida e dizia ``Você acha mesmo?''
Tirava a roupa e pulava de cabeça num lago muito frio.

Havia outros homens lá e todos achavam espirituosa a minha provocação
enquanto admiravam minha coragem.

Então nadei segura, na certeza de que nada disso seria interpretado como
um convite, e que ficar nua ou demonstrar minha força não me punha em
risco.

Quando saí, um dos homens presentes veio atrás de mim, mas porque tinha
ficado encantado com meu senso de humor, com minha força, com meu
desnudamento. O desafio e o despudor tinham sido para ele extremamente
eróticos. Saímos conversando. Eu caminhava nua.
