\chapterspecial{De caso pensado}{}{Silvio Rosa Filho}


\noindent\dotfill

\noindent Destinatário: \versal{Ao SÍLVIO DE ANNA P}.

\noindent Remetente: \versal{SÍLVIO ROSA FILHO}
\bigskip


\hfill São Paulo, 07 de setembro de 2017.
\medskip

\noindent Xará:
\smallskip

\noindent Ao longo desses dias estranhos que antes eram chamados de ``semana da
pátria'', não estive exatamente de folga, mas tentando diminuir o
acúmulo de mensagens não respondidas na caixa postal do computador,
anotando teses e dissertações acadêmicas, redigindo pareceres sobre
relatórios de pesquisa etc. etc. Cavei um tempo também para outras
leituras e me veio de repente o impulso de parar tudo e lhe escrever,
hoje, uma carta. Nada demais, reconheço. Apenas, algo condizente com
meus propósitos de testar, num dia feriado, certas condições atuais de
impossibilidade. Impossibilidade de uma leitura que, sem temer o
pedantismo, fosse leitura meditanda. Por isso tenho de ir logo avisando:
não serei brevíssimo como em correios eletrônicos, não serei breve como
em cartas que não passam de bilhetes aumentados. Serei o que tiver de
ser, ousando contar com sua boa vontade.

O que preciso lhe dizer inicialmente, Xará, é o seguinte: quando cheguei
ao fim do primeiro livro de Anna P., \emph{Tudo que pensei mas não falei
na noite passada} , você deixou de ser um mero equivalente da letra
\emph{S} na agenda escorregadia de Anna. Personagem mais que secundário
no rol das aventuras libidinosas da protagonista, você ficou mais para
uma espécie de holograma com nome social. E seguiu avançando e me
acompanhando pelas iniciais do abecedário erótico, até que, chegado ao
\emph{X} da minha questão, acabou por se transformar em alguém que
inspira e expira, pensa por conta própria e me interpela. A despeito de
uma intimidade que eu não pretenderia forçada, você um dia talvez me
responda fazendo uso do mesmo nome que é, devo admitir, um tanto
abrupto: o nome de Xará.

Pois eu devorei o primeiro livro de Anna P. E vi que era bom. Na
verdade, porém, restava uma questão de tato que, envolvida na equação,
me compelia a soletrar vagarosamente tudo aquilo que, durante a leitura,
fui dizendo a mim mesmo. Retomei o livro desde a primeira página, deixei
o cigarro fazer as vezes de incenso no cinzeiro e minutos depois --
leitura interrompida aqui, anotações marginais ali -- fui me permitindo
devanear pela tarde afora.

Pensei primeiro que fosse remeter uma cópia da carta a nossa Autora:
busquei palavras amenas, um arranjo mais ou menos elegante de argumento,
cheguei a ensaiar viravoltas, premeditando o epílogo. Iria redigir um
tipo de correspondência, de gênero clandestino, o andamento leve, o
subtipo esguio. Pensar, eu pensei. Buscar, eu busquei. -- Mas Xará, a
coisa está tão brava, de tão brava a coisa está, que é o caso de
escrevê-la com iniciais maiúsculas. E quem sabe? Escrevê-La com
\emph{K}: Koisa feia.

Você ficou de amores e sexo cru com a protagonista. Anna Publicada, a do
primeiro livro, não foi a única a ter notícias suas. O Zelito que também
aparece no Z do abc oral e que descobri ser nosso amigo em comum, me
contou umas coisas. A confirmar. ``Seu Xará'' -- me disse o Zelito --
``está definhando a olhos vistos, nem parece que já foi gente de carne e
osso''. Você sabe que Zelito é dado a lirismos nestes tempos incertos, e
também me escreveu algo mais ou menos assim: que você andou se remoendo
pelos cantos e recantos de seu quarto de dormir, engruvinhado em dobras
e redobras de lençol, amoldando lembranças do corpo dela em aromas de
travesseiro e concavidades de colchão. Para ele, que tem bom coração e
não falaria por mal, o pior é que você deu para inventar provérbios e
sentenças, frases e acrósticos tais como o ciúme é isso, a inveja é
aquilo; amor é isto, sexo é aquilo. Zelito ainda atribuiu a você uma
variação freudiana que o deixou inquieto, algo do tipo: ali onde o Isto
é, Aquilo há de ser. Ele acha que no fundo se trata, de sua parte, de um
pedido de socorro. Em suma, uma coletânea de momentos desesperados que
não deveriam ser lidos em termos de autoajuda, mas como uma tentativa de
você se equilibrar à beira do precipício, livrar-se de tendências
românticas para um dia deixar de viver à margem de Anna, a reeditada.

Terminei de reler \emph{Tudo que pensei} e tenciono me ater às
preliminares do segundo livro de Anna P., \emph{Do amor e outras
brutalidades}. Você pode supor que prefiro o Xará presente nas idas e
vindas de minha lembrança ao Sílvio discreto que comparece -- eu contei
três vezes -- ali no primeiro livro. Pois é a esse Xará que acontecem as
coisas, não ao Sílvio. Xará caminha por São Paulo e se demora à saída da
Biblioteca Mario de Andrade. Se não nos induzir ao erro a parcialidade
um tanto exibicionista de André -- outro nome do abc a quem Sílvio não
foi apresentado --, Xará gosta mesmo é dos relógios de Berta Dunkel,
trata de situar-se nos fusos históricos de Paulo Arantes e aprecia as
liberdades que Gilberto Tedeia toma com a tradição crítica brasileira.
Pelo visto André discorda do Zelito no que concerne ao ponto do
romantismo fora de hora, fazendo questão de acrescentar, ao seu gosto,
uma dose diária das \emph{Minima moralia}, a superação dos impasses
entre a prosa do mundo e a poesia do coração, uma frequentação assídua
das páginas de \emph{Claro enigma}.

Nessas questões de gosto, não sei se é Zelito ou André que tem razão.
Sei que Yan -- outro de nossos amigos comuns, um amigo de facebook --
ficaria espantado se acaso você não partilhasse tais preferências com o
Sílvio estrito de Anna P. Suspeito que ele enxergue tudo isso que lhe
digo de um modo menos altissonante do que poderá aparecer por escrito
nestas linhas; ele leva menos a sério aqueles nomes todos, tanto quanto
as partículas litigantes que elas podem ou devem significar. Isso de som
e de fonemas, distinguindo palavra e palavra, pessoa e pessoa, é um
território instável. Yan, lá em sua página digital, figura ser mais
brincalhão, escalando suas próprias qualidades abecedadas de ator,
bibliotecário, cenógrafo, dramaturgo e assim por diante. É um cara da
viração. Mas não se reconheceria, por exemplo, no antigo rótulo de um
factótum. Temos nossas distâncias, como você pode notar, mas friso que
tamanha distância não basta para me justificar.

Quem me dera ter a certeza de que você, Xará, tendo sobrevivido ao
primeiro livro de Anna P. e estando ausente do segundo, num terceiro a
moça talvez reserve, a você, hora, lugar e vez.

Bruno de vez em quando me telefona para dar notícias frescas a seu
respeito. Ele tem, na ponta da língua, afirmações sobre o seu paradeiro,
e um que das certezas dele me diverte, outras, francamente, me deixam
encabulado. Afirma que você nunca pôs os pés em Rio das Voltas, que
nunca se dá por achado em \emph{Do amor}. Que de fato você anda dando
voltas ao mundo, que costuma passear por uma Pompeia distante e
submersa, ali onde haveria lugar para amores novinhos em folha, cupidos
que teriam permanecido no estado de arquitetura em degelo. Por montes e
vales você foge ao rio de lava, escapulindo ao infortúnio de ver os pés,
panturrilhas e joelhos petrificados. Bruno assegura que depois de muito
vagar pelas encostas de Herculano ou Estábia, Nucéria ou Oplantis, você
queda hospedado no tempo latente de algum sítio arqueológico.

Não sei se convém insistir neste ponto. Houve noites em que Bruno ouviu
Anna chamando por você, Xará, sonhos atribulados dela. Para ele, que não
tem escrúpulos em exibir os seus delírios em público, a sina de Anna P.
alcança muitos homens. Seja. Quero crer que Bruno esteja dizendo: Anna,
essa mulher vulcânica, contribui para libertar leitores (e leitoras) da
saga insana de se tornarem homens. No caso, ela iluminou brasas frias da
desilusão em família, apostando que suas erupções vão fecundar o
solilóquio de quem a lê. Não imaginava ou não deu importância ao fato de
que tais leitores -- segunda edição esgotada -- entrariam em diálogo.
Acho que por enquanto só o Bruno tem olhos para ver você, um Xará sempre
insone, divagando pelas ruas e vielas de Pompeia, mergulhado numa noite
mais noturna que toda noite. Você, em meio a gente desperta pelo cheiro
de enxofre, os banhistas surpreendidos nas termas, as nuvens letais, a
chuva de cinzas finas e pedregulhos chamuscados. Pois Anna escreveu umas
coisas luminosas. Como um vislumbre que, começando nos cômodos escuros
da casa, depois ressurge pelos móveis em seus pontos de incandescência.
Depois vem a noite de novo. E tudo é noite.

Devido ao que vai por estes tempos, os nossos, eu não gostaria de
extravagar em guetos de lirismo. Bruno sublinha, contudo, essas vidas
paralelas de Sílvio, à margem do segundo livro. Da última vez em que ele
me escreveu, foi logo digitando que você colecionou epigramas coligidos
ao longo da Via Venérea. E emenda algo do seguinte teor: junto ao topo
do Monte Vesúvio, bem na borda da cratera, Sílvio aprendeu a enxergar,
não o abismo, mas um lago; em volta, não um ateliê de lápides
inconclusas, mas pedaços de templo e estátua, tomados a modelo vivo. Não
digo que não. Mas se tudo se verter um sintoma, o diagnóstico não
implica o Bruno, unicamente, mas cada um de nós, Xará. Para muito além
de Beltrano, inclusive, a sintomatologia compromete o \emph{B} que
aparece e reaparece no segundo livro. Como você já sabe mas não custa
lembrar, o que aconteceu entre ambos -- Anna e \emph{B} -- é isso o que
ela conta em \emph{Do amor e outras brutalidades}.

Passemos a \emph{Do amor}. Anna se apresenta com o recuo de quem tomou
nota de tudo e foi colhendo a dedo barbaridades constitutivas da vida em
família. São dois ou três episódios decisivos num conflito conjugal: os
desenganos, picantes, estão repletos de surpresa íntima e um delicado
suspense. São cenários onde o celular é um aparelho mutante, ora chicote
que fustiga à longa distância, ora queixada que acaba por matar o que
estava morto e defunto. Quem me fala desse jeito é o Caio. Anna delimita
as separações latentes de um casal comprimido na comunicação
troca-a-troca, nas negociações virtuais e numa separação manifesta, tudo
se preparando para contrastar, vivamente, com as conversas de verdade
que Anna terá com o filho, Quim. Concordo com Fábio quando assinala que
essas últimas conversas vão dar no tempo mais vivo do livro. É o dom
presencial de uma mãe, outrora tão brasileira, às voltas com seu bebê,
por vezes tão aparentado aos moldes de um reuso winnicotiano. Tales, não
o fisiólogo das águas primordiais, Tales Ab'Saber explica. Aqui sou eu
quem pode lhe assegurar: o filósofo da psicanálise aceitará convite para
sentar-se conosco, à mesa de bar, manter conversa amena sobre a livre
associação de ideias ou passar a conversa mais intensa. Por exemplo,
variação de frases hegelianas em dias de folga: o crescimento de Quim é
a morte de todos os pais.

Mas Xará, deixo um tantinho de lado essas coisas profundérrimas para lhe
dizer que \emph{B} não condiz em nada com os anseios de Anna P. Exceto
por ecos remotos que contracenam de celular a celular, \emph{B} mal
distingue a dançarina da halterofilista, intervalos infinitesimais que
sequer por um segundo devolvem a Anna -- na íntegra -- o nome do pai. Cá
entre nós, suponho que você, mano com unhas eriçadas, colheria algum
fruto maduro ali. Não ficaria lambendo, sozinho, os braços do sofá.
Ficaria?

Por onde você pode ver que, também neste ponto, discordo de Zelito.
Assim como discordo de Bruno: Anna não permitiu que danças de
acasalamento fossem levando o corpo dela, nem o de Quim, para uma
Pompeia situada nas vésperas do desastre. Sem rapto de deuses ou
sequestro de demônios, vale dizer que na hora h, ela gritou pelo nome de
um homem: -- Beltrano! Acontece que o cara acabou diluído precisamente
nesse \emph{B} de Beltrano. Noite em que todos os gatos são pardos? Em
absoluto. Na impossibilidade de dizer todos estes homens em avançado
estado de petrificação, fala de todos e de nenhum, exprimindo a própria
impossibilidade da expressão. Eu agora careço é de um aparte. Tenho um
amigo e correspondente, Edu, rapaz soteropolitano, abastecido de letras
e de raro valor, que recorda um caso da vogal suarabáctica. É caso
particular de uma tentação a que você, Xará, não deveria sacrificar suas
frases. ``Abissolutamente''. E repare como o acento secundário, caindo
na sílaba do i , dá com outro timbre, bem mais claro do que na fala de
uma esposa que, como a Laura do primeiro livro, dissesse assim: ``fiquei
indiguinada'', ``não adimito''.

Anna, por seu turno, é mulher de muitos pppês. Nisso é que eu concordo
com Rafa e talvez discorde do Paulo, não obviamente o Arantes, mas
aquele que nas alcovas do primeiro livro adorava xingá-la de puta e,
numa carta desabusada, sugere em disparate que Anna troque de analista.
Nela, no entanto, o que cresce, aparece e se multiplica é a presença de
espírito. Há também, com recuo autobiográfico, inúmeros múltiplos de
zero que, na leitura cursiva, da esquerda para a direita e desta para
aquela, prometem uma nulidade nua e crua. Digamos -- desta feita com
Olavo e também com Otávio -- que todo o interesse do segundo livro
reside menos na nudez, e mais, no próprio desvelamento da vida: a vida,
tida por natural e familiar, foi se confinando em nada menos que nada.

Mas você então me perguntará: o que é que sobra de todos esses
aniquilamentos? Respondo com prazer: eu gosto sobretudo daqueles
momentos em que Anna não insiste na destruição (redundante) do familismo
à moda da casa, brasileira; aquando, antes, ela persevera na tentativa
de tirar um partido vivo nos membros desconjuntados da sobrevivência, a
dois ou a três. É por aí que vai também o depoimento de meu amigo
Gilberto -- de passagem: um subtipo de movimento que transparece nos
escombros, que está tão a par dos corpos quanto possível e que deixa
entrever uma Terra de Ninguém, em muito assemelhada a nosso habitat
atual.

Tamanho esforço de veracidade -- agora sou eu que me pergunto, Xará: --
terá algum efeito libertador? Você me dirá que essa pergunta se desdobra
em muitos planos dessa capacidade de invenção que nossa Autora não se
cansa de por à prova. Sim, à primeira vista, a invenção parece
despretensiosa, pois a superfície é bastante afeita a desabafos. Nosso
querido Vicente não lhe escreveu? O palpite dele é que a proximidade dos
remordimentos é excessiva e persegue um pouco a leitura. O que se
persegue porém -- e nisso discordo de nosso amigo portenho -- é sempre
outra coisa. Como no caso de uma Medeia deslocada, o primeiro plano é
sombra. Ou ainda: fantasma que é preciso exorcizar.

Por este viés, quero dar um desconto à nossa discordância com Bruno e
fazer jus à sua busca de precisão nos delírios. Intuitivamente é claro,
logo de início, Anna já é ocupante de outros patamares de obscenidade.
Frente à maneira rotinizada dos exibicionismos contemporâneos, ela
discrepa. Não escracha; entra em erupção. Não cospe; implode. Expõe ao
vexame, portanto, a persistente e comezinha inviabilidade da família
monogâmica patriarcal.

Não quero passar ao que chamei tempo vivo do segundo livro sem antes lhe
dizer que no primeiro livro Anna -- Anna afobada e como se tudo fosse
pra já -- passava de letra em letra, sem um nome que as vocalizasse.
Percorria uma sociedade anônima de homens para fazê-los atuar de pronto,
com força emotiva ou expressiva: consonantais, eles ingressavam numa
paisagem de pesadelo para que o leitor visse, interpretasse, revisse,
pensasse. Ex-homens. A brevidade da notação impunha à leitura um desejo,
legítimo, de desalienação; e um impulso, discutível, de passar adiante.
Mas o melhor a meu ver estava no desencontro desses ritmos: justamente,
na disritmia é que seria possível tomar pé e, com sorte, desenhar um
salto para fora da repetição no sempre igual. Já pelo segundo livro, a
vivacidade brota e brota diversa. No registro da sobrevida em escombros,
para falar como o Gilberto, parece encerrar-se algo de precioso, um não
sei que de gesta demorada, com pérola prometida e como tal descumprida
no grão de areia das conchas, das classes e das etnias. Não obstante,
fora da concha, muda-se o registro. É um parto, Xará. Você, a ser
verdade que andou mesmo lapidando adágios e apotegmas, poderia resumir
esse alinhavo de faunas específicas e involução generalizada: -- nicho
de parentes, ninho de serpentes.

Daí uma questão que, se não me engano, veio preparando em surdina o
tempo vivo do segundo livro. Questão de reversibilidades, Xará, que
formulo para quando você se dispuser a sair de casa e me responder:
haveria uma porção de antídoto a extrair dessa overdose de
envenenamentos cotidianos?

Tempo de virar a página. Sair do gabinete de objetos de desejo
customizado, dizer adeus a lacanagens, acenar, por que não?, a um Lacan
mais íntimo, Jacques sem fatalismos. Ingressar nos vestíbulos de Eros e
outras enormidades civilizadas. Se você quiser, tome nota também: mais
para o gozo na cura que para o gozo no adoecer.

Pelo que vou lhe dizer por esta carta abaixo, a página, Xará, pediria um
subtítulo. Hélio, o cuidadoso Hélio, sugeriria algo que evocasse o arco
de um niilismo às avessas. O que eu sinto agora, no lento entardecer
desse dia funesto, é isto: pulsação parabólica; nem grega nem romana;
hebraica. Botemos o subtítulo, simplesmente: Mães \& Filhos.

A despeito do muito ressentimento, o recurso ao bebê, o filho Quim como
elemento de composição, seria uma solução habitual e, também, de
facilidade. Assim é, contudo, somente para a frieza de quem vive a
repetir que a paternidade é incerta e, ao mesmo tempo, permanece
insensível à carga usual de responsabilidades e sobrecargas
contemporâneas de responsabilização. Estas últimas, sem dúvida, têm peso
e valor diversos, sobretudo num país de mães entregues à própria sorte.
Mostram, no entanto, uma Anna situada a anos luz de se tornar mamãe bem
comportada.

Valeria a pena então se demorar, Xará, nos instantes em que Anna escreve
a Beltrano e avisa: esteve todo esse tempo ``tentando falar as coisas
dentro da complexidade delas''. Anna P., ave rara, pensa enquanto
escreve. Se Zelito estiver certo no que diz de suas tendências e
inclinações à escrita lapidada, você bem que poderá retomar essas coisas
que só faço aqui rabiscar: depois das complicações da vida adulta, a
simplicidade alcançada na graça de uma criança; por onde se vê que
graças infantis não cancelam infantilidades sem graça.

No losango amoroso que traduzo para uso próprio, seria o caso de dar
vasão a duas ou três assertivas vigorosas. A combinação de autenticidade
materna e tolice marital é, certamente, um achado. Malgrado toda a busca
de energia, empenho na vida conjunta e engajamentos para uma metamorfose
das relações de interdependência, o que se transforma no curto-circuito
dos afetos são os próprios corpos em relação, ou melhor, as
intercorporeidades. Viram documentos de um estrato social subordinado
que, caligrafando a sua breve ocupação na cena contemporânea, é uma
espécie zoológica sob permanentes ameaças -- e chantagens -- de
extinção. Você diria, sem complicar mais do que o necessário: o mais
vivo no vivo é tomado ao vivo, instante que precede a morte ou a suposta
impossibilidade de conversão. Passemos agora a palavra a Anna P. quando
a moça fala na língua de conversa:

\begin{quote}
\emph{Voltei para a minha classe com o rabo entre as pernas}

Voltei com o rabo entre as pernas e uma criança de cinco meses no colo.

Voltei para o bairro onde eu nasci e voltei para a universidade.

Voltei para a família e para as amigas de antes.

Em nada me reconheço.
\end{quote}

A consequência mais palpável, Xará, será um extraordinário
passa-a-palavra à criança. Quando você se aproximar do final desse
segundo livro, não deixará de perceber que a intensidade dos cuidados
maternos transborda os domínios da normalidade pequeno-burguesa.
Buscando refúgio numa sabedoria em estado nascente, a composição toma o
partido desse movimento inevitável que leva o bebê ao quarto ou quinto
ano de vida. É toda uma pré-escola do desejo, pedagogia na qual a vida
vai mais longe do que a pedagogia. Sair do cativeiro das paixões
familiais, por essas e outras, conduz ao não se sabe onde. Porque ainda
não se sabe.

Tanto melhor para o livro. Por motivos que lhe são intrínsecos, Anna usa
mas não abusa daquilo que os clássicos chamavam de \emph{graça}, isto é,
beleza em movimento. Você retomará o fio dizendo que, de agora em
diante, não sabemos em que infortúnios vai Anna P. se aventurar. Mas é
certo que também neste passo os caminhos se bifurcam e não escondo minha
balança quando ela pende para os lados de Quim. De um lado Freud,
disfarçado em pensador da cultura, cochicha aos ouvidos da mãe:
\emph{His Majesty The Baby}. Mas ao mesmo tempo vemos Quim, o filho que
não precisa se eternizar em governanças de trono e altar: Quim se limita
a pintar o sete com os vocábulos que acaba de aprender, ou ainda, faz o
diabo com as palavras que apenas começa a inventar; tais flagrantes de
um narcisismo saudável na criança podem encher o leitor de comovidas
alegrias. De outro lado, Xará, é igualmente certo que, na própria
impossibilidade de uma tragédia brasileira, mano Bruno estará abastecido
de motivos quando disser ``-- É um gozo ler Anna: sua libido, à maneira
de Medeia, será mais forte que as coisas que ela quiser''.

Você já pôde perceber que não vou lhe pedir desculpas por essa carta
compridona. Se imaginar quanto me aborrece escrevê-la assim, no dia
oficial da independência! Você sabe quanto lhe quero bem e posso até não
lhe agradar com meu jogo de espelhos, mas, Xará, no apreço volátil pelo
diz-que-me-diz, para mim seria um verdadeiro deleite saber o que pensam
Denis, Ivan, Jaime, Kleber, Marcos, Thiago, Wilson. Porque, francamente,
são muitos e sugestivos os pontos de contato entre vocês, Anna e Quim.
Alguns serão mais misteriosos e outros mais explícitos, como os que
surgem do balbucio de leitores e leitoras conversando entre si. Por
essas e outras, espero que você me escreva, sem nenhuma pressa, mas
logo.

Para encerrar, preciso lhe dizer que Anna continua querendo, podendo,
gostando. Vicente toma gosto mesmo é no jugutear: diz que, tudo bem
sopesado nestas nossas conversas e desconversas, Anna voltou
latino-americanizada. Retomando o resultado à luz de \emph{Tudo o que
pensei} e do tisne um tanto vulcânico \emph{Do amor}, faço questão final
de lhe dizer: o resultado é, não um segundo livro, mas um díptico. Em
pinceladas vigorosas, esse díptico funde mosaico erótico, cio de animais
em sua humana animalidade, emboscadas da luta de classes. É uma
revelação condicional: se estivesse no nexo entre Eros \& Política, só
chegaria até nós por aproximações sucessivas: numa delicadeza pequenina,
pelo prisma de Quim.

No ``sonho erótico'', que põe ponto final ao segundo livro, Anna estará
com os pés suspensos, nua, entre os homens. Eu a vejo de costas,
dançarina, as mãos levantadas, girando sobre si mesma. Realizando, num
estilo de deformação que só o texto pode enformar, o seu desejo
propriamente dito. Pensando melhor, Xará, quando toda mudança parece uma
piora, o caso Anna significa que tem cabimento não temermos a cidade sem
nome. Ninguém terminou de voltar, ao menos até hoje, à sua própria
classe. E só agora percebo: se de fato sou eu quem escreve estas linhas,
é porque de algum modo nós já saímos, juntos, para fora daquelas
iniciais em que ficaram ancorados os nossos nomes.

Por hoje só.
\medskip

\hfill Sílvio
