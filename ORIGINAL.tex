%!TEX root=./LIVRO.tex
\epigraph*{A uma criança}

\paginabranca
\emph{Aprendi quase tudo sobre o amor. }

\emph{Toda ilusão vital, toda violência sem sentido.}

\emph{Por isso hoje amo quase em silêncio.}

Tales Ab'Saber

\part{Do amor e outras brutalidades}

\part{Antes de uma}

\chapter{O náufrago}

Ele trepava como um náufrago. Eu me agarrava a ele tentando não afundar.
``Eu te amo Anna'', ele dizia com a boca no meu ouvido e o pau na minha
buceta, ``Eu te amo, minha linda'', e metia mais forte, ``Te amo
muito'', e mais forte, ``Muito'', mais forte, ``Muito!'' Eu gritava de
prazer e desespero. ``Também te amo, meu amor. Eu só quero você, só
você, só você!'' O suor do rosto dele pingava em meus olhos e chorava a
lágrima mais salgada. ``Eu também só quero você, Anna!'' Eu bebia o suor
do seu pescoço e nos beijávamos sufocando meus gemidos. \emph{Ainda
vamos nos afogar, ainda vamos nos afogar...} Só tínhamos para beber a
saliva um do outro naquele quarto pobre de hotel. Ele gritava ``Eu te
amo, meu amor!'' e seu rosto sofria enquanto seu pau me estocava com
toda força. Eu perdia as palavras e respondia aos gritos. Eu me meu
náufrago morremos toda semana nos braços um do outro. Ele diz ter por
mim um amor maior que tudo; eu tenho casa, marido, família e filhos
planejados. E agora choro o suor dele, me afogo nos beijos dele, quero
morrer nos braços dele. Eu, que já beijei os olhos da morte, essa morte
que me é tão familiar, eu, que tinha planos, tinha uma família, agora
vejo tudo rodar à minha volta enquanto me afogo. Tudo vem à tona, mas
submirjo. Vejo através da água a minha vida na superfície: a casa
reformada e mobiliada com carinho, a cortina recém comprada ainda por
pendurar, o abajur para a mesa da sala. Vejo a luz do abajur através da
água. Vejo o rosto magoado do meu marido. Mas me afogo e afundo no suor
do meu náufrago até tudo escurecer. Ele me beija. ``Você é a mulher da
minha vida, Anna!'' A paixão me aprisiona. Vejo o juízo girar nas águas
da superfície. Dentro d'água é escuro e eu sufoco. Meu marido me beija e
eu sufoco. ``Fica comigo, Anna. Eu construo uma vida com você!'' Uma
vida submersa. Ele se agarra a mim como um náufrago, eu me agarro a ele
e submirjo.

\chapter{Mar}

\begin{verse} 
O poeta\\
Contempla o mar\\
E depois canta\\
Eu vou lá\\
Mergulho inteira\\
E depois conto\\
\end{verse} 
	
\chapter{Investimento}

\textbf{P.} 13 de julho de 2011 às 11:58

Saudade docê - acho que viciei

\textbf{B.} 13 de julho de 2011 às 11:59

Eu também tô com saudade amor

\textbf{P.} 13 de julho de 2011 às 12:13

A mancha de sangue não saiu do lençol....

\textbf{B.} 13 de julho de 2011 às 12:29

Nunca mais vai sair

\textbf{P.} 13 de julho de 2011 às 12:30

Rs

\textbf{B.} 13 de julho de 2011 às 13:15

Tô quase terminando o Kafka. Se ninguém me brochar eu termino hoje ainda

\textbf{P.} 13 de julho de 2011 às 13:16

Tá gostando?

\textbf{B.} 13 de julho de 2011 às 13:17

Estou criando uma admiração pelo Gregor

Que ao meu entender vai se foder do começo ao fim

Mas ele é foda

Muito louco

Um pouco alienado

Mas quem não é?

Sofremos como o Gregor o tempo todo

\chapter{No canto do estoque}

\textbf{B.} 14 de julho de 2011 às 19:49

Pô não aguento mais

Quero ir embora

Tô ficando louco aqui

\textbf{P.} 14 de julho de 2011 às 19:49

Tá com pouco trabalho?

\textbf{B.} 14 de julho de 2011 às 19:50

Trampei pra caralho até às seis

Agora tá foda

Tô sem nada pra fazer

Não dá pra ler porque tiram minha concentração

Tô aqui escondido no canto do estoque

\chapter{Ternura freestyle}

Puta

Canalha

Cachorro

Cadela

Patife

Piranha

Gostoso

Gostosa

Vagabundo

Vadia

Moleque

Biscate

Filho da puta

Gostosa

Gostoso

Moleca

Bandido

Sem vergonha

Ordinário

Safada

Indecente

Vulgar

Sórdido

Cachorra...

\chapter{Paixão}

Aquele papelzinho que vai para a máquina no bolso da calça sujando toda
roupa que você pensava ter lavado.

\chapter{Desconfiança}

\textbf{P.} 15 de novembro de 2011 às 12:08

Tô angustiada

\textbf{B.} 15 de novembro de 2011 às 12:08

Por quê?

\textbf{P.} 15 de novembro de 2011 às 12:08

Tô me sentindo sozinha nesse mundo

E triste com a nossa discussão

\textbf{B.} 15 de novembro de 2011 às 12:09

Anna eu te amo

Tô contigo. Agora, se você se sente sozinha mesmo assim, eu não sei bem
o que fazer

\textbf{P.} 15 de novembro de 2011 às 12:10

Eu queria que você parasse de desconfiar de mim. Entenda: eu não gosto e
não sei trair

\textbf{B.} 15 de novembro de 2011 às 12:10

Já parei

\textbf{P.} 15 de novembro de 2011 às 12:11

Eu queria te dizer que fiquei nove anos em uma relação fechada e nunca
traí

Depois fiquei cinco em uma aberta e também não traí, porque era aberta

Agora eu tô com você e é isso

Mas é chato ficar pisando em ovos

\textbf{B.} 15 de novembro de 2011 às 12:13

Concordo Anna

Vou cuidar de mim

\textbf{P.} 15 de novembro de 2011 às 12:13

Eu PRECISO que você faça isso

Não deixa a neurose vencer o amor

\textbf{B.} 15 de novembro de 2011 às 12:16

Vou me policiar

Vou depositar minha confiança em você

Aí é contigo

Faz o que for melhor

Você e livre

Eu não sou você

\textbf{P.} 15 de novembro de 2011 às 12:20

Sou livre, por isso sou mais responsável que a maioria

É isso que eu quero que você entenda

\textbf{B.} 15 de novembro de 2011 às 12:25

Então

Fica em suas mãos meu amor

Minha confiança é sua

\textbf{P.} 15 de novembro de 2011 às 12:26

E nas suas mãos também, porque eu também confio em você

\textbf{B.} 15 de novembro de 2011 às 12:34

Vou lutar por você

E ver o que a vida me reserva

É o que me resta

Estou cansado demais

Só estou com muito medo de me machucar

\textbf{P.} 15 de novembro de 2011 às 12:36

Seu ciúme me joga num lugar horrível

Ciúme não é amor

\chapter{Soneto submerso}

Meu náufrago tem olhos frágeis

Doídos, cílios compridos

Um procura algum sentido

O outro sente, impotente

Meu náufrago me submerge

E quero levá-lo à tona

Mas usa pedras nos pés

E vive na subzona

Respiro água faz tempo

Ando no fundo do mar

Com as narinas tapadas

E os olhos marejados

Porque não basta o amor

No abraço dos afogados

\chapter{Vagabunda}

\textbf{B.} 23 de dezembro de 2011 às 11:07

Desculpa

\textbf{P.} 23 de dezembro de 2011 às 11:07

Quero desculpar

Mas quem desculpa é o tempo e as mudanças de atitude

\textbf{B.} 23 de dezembro de 2011 às 11:41

Eu só quero seu bem

\textbf{P.} 23 de dezembro de 2011 às 11:41

Mas está me fazendo mal

\textbf{B.} 23 de dezembro de 2011 às 11:42

Eu não pensei mal de você, eu só disse o que pensava, mas

sem querer te magoar

Só não quero que você seja uma mulher malvista. Não por você, mas pelos
outros

\textbf{P.} 23 de dezembro de 2011 às 11:43

Eu sempre vou ser mal vista, eu sou mulher

\textbf{B.} 23 de dezembro de 2011 às 11:46

Então meu amor, aí que tá. Seja mais escrota com quem é realmente
escroto contigo

Você tá entendendo?

\textbf{P.} 23 de dezembro de 2011 às 11:48

Eu entendo, só não concordo

Tô muito cansada. Vou deitar um pouco, depois a gente conversa

\textbf{B.} 23 de dezembro de 2011 às 18:53

Tá tudo bem contigo?

\textbf{P.} 23 de dezembro de 2011 às 18:53

Não. Só durmo e choro

\textbf{B.} 23 de dezembro de 2011 às 18:54

Fala então

\textbf{P.} 23 de dezembro de 2011 às 18:54

Não me saem da cabeça as palavras que você usou

\chapter{Vagabunda}

Tem palavras que a gente não deve usar nunca

\chapter{Puta}

Ninguém mais leva a sério

O seu vitupério

\chapter{Não vai dar certo}

Mas pode ser

que agora

eu prefira o caminho errado

ao correto

\chapter{Casamento}

Morar junto é neurótico

E antierótico

\chapter{Ela gosta é dos coitadinhos}

Gosto de homens que trepam para hoje, porque amanhã ninguém sabe o que
será. Difícil encontrar isso na classe média. A classe média tem futuro.
Só encontrei entre pessoas que vieram das periferias e refugiadas. Um
dia os amigos do Beltrano disseram para ele:

\chapter{Foda}

O foda é que fui a foda da vida de muita gente

Mas isso não me fez menos fodida

\chapter{Lúmpem}

Lúmpem, margem de manobra, massa acrítica, herança escravista,
inorgânico, proletariado excedente, trabalhador informal, inimpregável,
pobre coitado, nem nem, subproletariado, sobrepopulação trabalhadora
superempobrecida permanente, não qualificado para o mercado de trabalho,
classe em si, despolitizado, precarizado, da horda dos ressentidos, para
lá da ponte, vagabundo nato, moldado para o encarceramento, ou
extermínio, sobrevivente, maconheiro, zé ninguém, fulano, Beltrano.

\chapter{Os intensos se atraem}

E se destroem

\chapter{Dupla jornada}

(Junho de 2013)

Congelei. É isso. Congelei. Não acredito. Não posso me mexer. No
entanto, algo se move dentro da minha estagnação. Não é possível. Não é
possível que qualquer coisa se produza dentro de tamanha paralisia. Mas
está acontecendo. De que forma isso se justifica? Estou sozinha nisso?
Alguém vai me ajudar? De forma alguma. Estou sozinha. Ninguém. Nada.
Pessoas continuam nascendo na catástrofe. Coisas também. A vida
melhorou. E ainda assim é insuportável. Minha mãe nasceu no nazismo. Eu
nasci na ditadura. Queria ter gerado bem antes. Mas o que se criaria nos
últimos vinte anos? Agora essa democracia. Tolerância ao intolerável.
Falta de criatividade. Sobrevivência anestésica. Felicidade é não doer.
Da minha janela vejo o beco. Beco não, cortiço. Tudo me dói. Tudo me
dói. Não consigo me mover. Ninguém vai me ajudar? Ninguém vai a lugar
nenhum. Ninguém se desloca. Nem se o outro estiver morrendo. Não doer é
minha maior ambição. Foda-se a felicidade. Preciso me alimentar. E não
me movo. Será que morri enquanto sobrevivia? Estou criando ou morrendo?
Estou reproduzindo porque não me matei? Alguém vai chegar aqui? Há
alguém entre mim e o cortiço? Tenho fome. E se eu morro? Nada nasce? Não
tenho certeza se quero não poder morrer mais. E se a vida for inviável?
O que vai ser do que engendrei? Estou paralisada. Como faço para comer?
Preciso me levantar. Preciso me mover. Estou enlouquecendo. Ouço vozes.
Não tem ninguém. Ouço vozes. Vejo coisas. Vejo os ratos deixando os
cortiços e se tornando pessoas. Ou ao menos parecendo. Vejo pessoas se
movendo. Estão na rua. Uma só voz. Creio que sonho, mas é bonito e me
alimenta. Já posso me sentar. Não estou na rua, mas me movo. Vozes e
pessoas em movimento. Algo se desloca. Agora posso conceber. Mesmo sem
saber o que se produz. Ou reproduz. Mesmo que seja um devaneio. Algo já
pode nascer. Inclusive nada. Mas agora é possível.



\part{Uma cria}

\chapter{Esgotamento}

\textbf{P.} 22 de janeiro de 2014 às 13:51

Eu tô doente de stress e você não tá levando a sério. Uma tá sofrendo
porque a mãe e o pai não deram conta de cuidar dela. Eu tô tendo crises
de choro e de ansiedade direto e tô fazendo um esforço enorme pra
arrumar essa bagunça toda. Então, por favor, não me ligue em tom de
cobrança.

Eu tô aqui pra cuidar da nossa cria (e de mim, pra conseguir cuidar
dela)

\textbf{B.} 22 de janeiro de 2014 às 13:51

Anna

Tô te cobrando nada não

Tô preocupado com vocês

\textbf{P.} 22 de janeiro de 2014 às 13:52

Eu não tô te ligando porque sai caro

A gente se fala por mensagem ou por aqui

\textbf{B.} 22 de janeiro de 2014 às 13:54

Tô com saudade dela

\textbf{P.} 22 de janeiro de 2014 às 13:55

Eu também tô triste que vocês estão longe

Mas teve que ser assim

\textbf{B}. 22 de janeiro de 2014 às 13:56

vai ficar até quando?

\textbf{P.} 22 de janeiro de 2014 às 13:56

Não sei, B.

Até ficarmos bem

\textbf{B.} 22 de janeiro de 2014 às 14:01

Uma tá mamando bastante

Fico preocupado com o tanto de mamada que ela dá por dia

Isso tá te sugando

\textbf{P.} 22 de janeiro de 2014 às 14:04

Mas a falta de sono suga muito mais

Positive Vibration

Tive de descobrir na prática que não é a potência da droga que determina
o grau de dependência -- e sim a falta de perspectivas. E quando essa
ausência vem revestida de um estiloso rastafári, boa música e infinitos
rituais coletivos de Jah, fica muito difícil encarar como vício. Daí
para a frente, qualquer fresta de possibilidade será desfibrada,
enrolada, tragada e asfixiada pela dependência. E cumpre-se o círculo
redundante.

Toda metade masculina de Rio das Voltas vivia anestesiada assim. Jovens
deprimidos em festa. Deprimidos na festa do sol.

\chapter{Recuperação}

\textbf{B.} 24 de janeiro de 2014 às 12:58

Tá tudo bem entre a gente Anna?

\textbf{P.} 24 de janeiro de 2014 às 13:17

B., vou falar meio interrompido por causa da Uma

Eu tô chateada de ter chegado a esse ponto de cansaço físico e mental

De ter vindo pedir penico pra minha mãe

E sobretudo de ver que nossa filha não estava bem

Acho que falhamos, eu e você

Como casal e como pais

E estou tentando me recuperar disso também

\textbf{B.} 24 de janeiro de 2014 às 13:19

Anna

Não tem falha alguma

Nossa filha vai ficar bem

\textbf{P.} 24 de janeiro de 2014 às 13:20

Vai ficar, mas não estava

E eu também não estou

E você também não

\textbf{B.} 24 de janeiro de 2014 às 13:22

Anna quem tá doente é você

Uma tá sentindo o reflexo disso

E agora não é hora pra neurose de ter jogado a toalha

Força, mulher

\chapter{Aquela grana}

\textbf{B.} 24 de janeiro de 2014 às 13:28

Anna você sabe quando volta?

\textbf{P.} 24 de janeiro de 2014 às 13:33

Ainda não tenho ideia

\textbf{B.} 24 de janeiro de 2014 às 13:34

Então meio chato falar

Mas aquela grana deve durar só até esse fim de semana

Tudo bem se você mandar algo?

\chapter{Vinte quilos}

\textbf{B.} 27 de janeiro de 2014 às 10:21

Como você tá?

\textbf{P.} 27 de janeiro de 2014 às 10:50

Ainda no osso

E perdendo um quilo por semana

Já foram 15

\textbf{B.} 27 de janeiro de 2014 às 10:52

Mas por qual motivo?

Saúde?

Ou Uma tá mamando muito?

\textbf{P.} 27 de janeiro de 2014 às 10:52

Acho que os dois

Porque o cansaço me tira o apetite

Então tô comendo por obrigação

E meu estômago encolheu muito

Uma tá começando a mamar menos

Essa noite já dormiu um pouco mais entre as mamadas

Ela já tá feliz de novo

\chapter{Mais cinquenta}

\textbf{B.} 29 de janeiro de 2014 às 10:00

Você se incomoda de transferir 50 reais pra conta do Jonathan?

Porque eu tava duro

E ele me emprestou pra compra rango aqui

No fim de semana

\chapter{Retomada}

\textbf{B.} 29 de janeiro de 2014 às 21:48

Me atende

Anna só atende por favor

Eu não Tô legal

Olha eu não sei o que tá rolando

Mas eu só queria ouvir sua voz

Ia me fazer bem agora

Eu tô numa puta crise de ansiedade

Só precisava falar contigo um pouco

Só pra me sentir mais seguro

\textbf{P.} 30 de janeiro de 2014 às 07:15

Você enlouqueceu?

\textbf{B.} 30 de janeiro de 2014 às 08:36

Anna

Tive uma noite horrível ontem

Mas tô tentando melhorar

Me desculpa

Errei

Foi desespero

Só tive uma noite ruim

\textbf{P.} 30 de janeiro de 2014 às 08:57

Eu tô tendo noites horríveis desde que Uma nasceu e pouquíssima
solidariedade sua

Porque no fundo você acha que a responsabilidade de cuidar da cria é só
minha

\textbf{B.} 30 de janeiro de 2014 às 09:04

Que isso?

De onde você tirou isso?

\textbf{P.} 30 de janeiro de 2014 às 09:04

Você não consegue entender quando eu digo que não aguento mais

\textbf{B.} 30 de janeiro de 2014 às 09:04

Nunca foi assim

Você não tá com a cabeça legal

Eu também não

Mas a gente tá nessa junto

\textbf{P.} 30 de janeiro de 2014 às 09:05

Você não toma iniciativa de procurar soluções, pediatra, livros, tentar
fazer ela dormir sem precisar de mim

Dormir mais cedo pra acordar melhor pra cuidar dela

Você acha que eu sempre posso aguentar mais e se responsabiliza muito
pouco

\textbf{B.} 30 de janeiro de 2014 às 09:06

Tenho te falado coisas boas

Dando demonstração de afeto e carinho

Te apoiando aí

\textbf{P.} 30 de janeiro de 2014 às 09:06

Só que não era pra eu ter precisado vir

\textbf{B.} 30 de janeiro de 2014 às 09:07

Não era mesmo

Mas você foi

\textbf{P.} 30 de janeiro de 2014 às 09:07

Porque você não me deu suporte suficiente!

\textbf{B.} 30 de janeiro de 2014 às 09:07

O queee?

\textbf{P.} 30 de janeiro de 2014 às 09:07

Segurar o bebê pra eu poder mijar e tomar banho não é suficiente

\textbf{B.} 30 de janeiro de 2014 às 09:07

Para de falar comigo como se eu fosse um filho da puta

Não faz isso

Eu tô dando minha vida pra você

\textbf{P.} 30 de janeiro de 2014 às 09:08

Eu corri sozinha atrás de pediatra, diarista, alimentação, lembrar
vacinas, fazer a bolsa dele pra sair, tudo

Fui eu que mudei minha vida toda pra você poder se erguer

\textbf{B.} 30 de janeiro de 2014 às 09:09

Para de falar desse jeito

Você vai chegar onde eu não quero

\textbf{P.} 30 de janeiro de 2014 às 09:09

Passei do meu limite

Uma tá perigando ficar sem mãe

E agora?

\chapter{Rompimento}

\textbf{P.} 31 de janeiro de 2014 às 07:10

B., Meu amor

Eu sinto sua falta o tempo todo

Falta dos momentos em que a gente conseguia rir juntos, cada vez mais
raros, mas tão bons!

Falta do seu olhar de amor pra mim

Do seu cheiro quando eu te abraçava na cama

Do sexo indescritível que só a gente faz

Eu deito todo dia naquela cama e lembro do amor maravilhoso que a gente
fez quando eu me separei

Seu sorriso de amor e carinho olhando pra mim

A entrega dos nossos corpos

A alegria de estar juntos

Acho que posso dizer que você foi a maior paixão da minha vida

Por você e pela gente eu fiz coisas que não fui capaz de fazer por mais
ninguém

Eu investi o que eu podia e o que eu não podia na sua profissionalização
porque sempre admirei sua inteligência e acreditava - ainda acredito -
que você só precisava de uma oportunidade

Você, do seu lado, enfrentou os colegas que, mesmo sendo um bando de
medíocres, deve ter sido muito difícil enfrentar

Depois topamos fazer um filho, mesmo sabendo que a hora não era boa nem
pra mim nem pra você

E Uma tá aí, e você sempre vai ser o pai dela e o cara que teve a
coragem de entrar nessa comigo

Te admiro por isso também

(Tenho que atender o bebê, depois continuo)

\textbf{P.} 31 de janeiro de 2014 às 08:53

Nós dois, em função desse grande amor, enfrentamos família e amigos a
ponto de perder nossas referências no mundo. Você saiu do lugar que você
ocupava, bom ou ruim, pra fundar um novo lugar comigo. E eu também
deixei um mundo no qual eu, mal ou bem, me localizava, pra construir
outra vida com você.

E essa é a nossa história de amor, ela é linda. Ela é tão linda que só
de pensar em terminar eu não consigo mais comer nem ficar sossegada e tô
definhando dois quilos por semana aqui.

Não, querido, minha situação não tem nada de confortável. E no meio
desse sofrimento enorme eu ainda tenho que me virar do avesso pra não
deixar nossa cria sofrer.

Tá difícil, dá vontade de morrer só pra poder descansar um pouquinho,
mas não vou me jogar de lugar nenhum. Eu já não tenho esse direito.

(Deixa eu tentar deitar o bebê)

\textbf{B.} 31 de janeiro de 2014 às 09:19

Você vai continuar comigo?

\textbf{P.} 31 de janeiro de 2014 às 09:36

Amor, deixa eu dizer tudo

Nossa história de amor é incrível do ponto de vista de tudo o que
enfrentamos pra ficar juntos. E enfrentamos com muita coragem.

Mas há a parte em que falhamos, falhamos miseravelmente. Falhamos um com
o outro, como casal, como homem e mulher. E o mais doído é que essa
parte talvez nem tenha a ver com as dificuldades externas que tivemos
que enfrentar.

\textbf{B.} 31 de janeiro de 2014 às 09:39

Atende aí

\textbf{P.} 31 de janeiro de 2014 às 09:40

Não posso atender que Uma dormindo aqui

Deixa eu falar, depois falamos por telefone

\textbf{P.} 31 de janeiro de 2014 às 09:44

Vou continuar escrevendo porque esse é o tempo que eu tenho

Daqui a pouco ela acorda

Eu preciso falar da parte em que falhamos

A gente não pode esquecer que mesmo com tanto amor, estamos pra terminar
desde que começamos

Primeiro era a minha separação

Depois as inseguranças de falta de grana e o ciúme que muitas vezes me
machucou e ofendeu - e te machucou e ofendeu também

Nossa convivência era feita de um afeto imenso, mas de muitas brigas
também

Me entristecia tanto computador e tão pouca atividade juntos em casa, me
entristecia a gente precisar de maconha pra rir juntos de forma um pouco
mais descontraída. E, vamos admitir, talvez a maconha estivesse lá pra
compensar uma grande falta de afinidade.

Acho que você tem amigos bons e gentis, mas eu não me divirto com eles.
Faltam coisas em comum. Da mesma forma, sei que você gosta dos meus
poucos amigos, mas também falta afinidade pra conviver mais de perto com
eles. Infelizmente, o amor não supera isso.

Falhamos muito na convivência, eu me ressentindo do seu mau-humor, da
falta de empenho em deixar o ambiente mais leve dentro de casa; você se
ressentindo da minha neurose de sofredora, da minha incapacidade de me
desvincular de quem me faz mal. E nessa história ninguém é inocente, nem
exatamente culpado.

E a toda hora, em todos esses três anos, a vida a dois foi difícil, não
vamos mentir. E a gente nunca deixou de falar em separação, como nunca
deixamos de desejar construir uma vida juntos.

Eu acho que a gente se fortaleceu muito, você me amando, se dedicando;
eu te amando, te incentivando. Mas também nos enfraquecemos muito, cada
um querendo que o outro fosse um pouquinho mais fácil, um pouco
diferente, um pouco menos difícil de lidar.

Até que, no final de 2012, depois de dois anos juntos e insuficientes,
estávamos esgotados dessa luta, estávamos desistindo. E fomos viajar e
foi ainda pior. Não só porque entrou o meu lixo familiar na história, o
que não foi pouco, mas porque nós já não estávamos conseguindo ficar
bem. Era cerveja de dia, vodca até a embriaguez à noite. Eu, magoada, já
não conseguia me entregar na cama. E você, não sem razão, se ressentia
disso. E tinha certeza que nós íamos terminar depois dessa viagem.

Só que eu engravidei. E então renovamos nossa fé na marra mesmo e
partimos pra mais uma jornada que, infelizmente, foi um pesadelo.

Dessa parte não vou conseguir falar muito porque dói demais ainda. E
quase morremos as duas, eu e Uma, de abandono, de tristeza, mas
sobrevivemos.

Quando Uma nasceu, minha vida estava arruinada - está ainda. Eu não
tinha mais doutorado, nem emprego, nem forças pra voltar pro rolo
compressor das escolas particulares. Estávamos sozinhos com uma criança,
perdidos, sem saber o que fazer.

A decisão de ir pra Rio das Voltas não foi exatamente um desejo, nem meu
nem seu. Lá tinha a esperança de algum apoio familiar, mas sobretudo eu
tinha a certeza de que aí era o melhor lugar pra você começar sua vida
profissional.

Eu também acreditei que poderia me inserir em alguma faculdade, mas acho
que nisso errei. Pelo visto a cidade aí só funciona na base do favor de
algum "político" e isso nunca vai me beneficiar.

(Preciso empurrar alguma comida, depois continuo)

\textbf{B.} 31 de janeiro de 2014 às 10:41

Anna fala comigo

Só preciso saber que decisão você tomou diante disso tudo

Não preciso passar por isso

Eu tô mal aqui

\textbf{P.} 31 de janeiro de 2014 às 11:10

B.

Eu tô tentando falar as coisas direito, com delicadeza, dentro da
complexidade delas. E tô pensando enquanto escrevo.

Mas como não sei quando vou poder continuar a escrever, vou falar o que
você quer.

Eu não quero voltar pra vida que a gente tava levando aí.

Também não quero ficar aqui, mas não tô vendo outro jeito agora.

Nossa vida tava escura, cheia de brigas e ataques desde o processo de
mudança.

Mas a gente teve fé e foi em frente mesmo com nossos atritos avisando
que era melhor parar e pensar.

Mas a vida aí, numa casa escura, cheia de brigas, em uma cidade estranha
e sem conhecer ninguém, e com um recém nascido... eu não dou conta

Você pode acreditar que é covardia minha, tudo bem.

Mas lembra que quando não era eu que ameaçava ir embora, era você. Que
muitas vezes você também disse que não queria mais. E você não queria
mesmo, porque a vida tava ruim demais e ninguém tem que querer isso.

A minha tortura psicológica, desde sempre, é a ideia de afastar Uma do
pai e da irmã. B., eu fui até onde eu podia, mas quando ela começou a
ficar mal, a ficar insegura e sem mãe, então não deu mais pra vacilar na
decisão.

Eu tô destruída com tudo isso e ainda por cima não deixei de te amar.

\textbf{B.} 31 de janeiro de 2014 às 11:20

Anna, você tá se separando de mim mesmo?

Sério?

É isso mesmo que tá acontecendo?

\textbf{P.} 31 de janeiro de 2014 às 11:21

Estou, assim como você também quase foi embora várias vezes, agora eu
estou indo. E é terrível tudo isso, mas é isso mesmo que está
acontecendo.

\chapter{Tanta gente me chama}

de musa

de deusa

de fusa

Mas não me ama

\chapter{Eu só queria o seu bem, meu bem}

\textbf{B.} 31 de janeiro de 2014 às 11:21

Anna só atende a última vez o telefone aí

Eu vou aí pra ver Uma

E resolver isso contigo direito

Jamais pensei que você ia fazer essa molecagem comigo

\textbf{P.} 31 de janeiro de 2014 às 11:49

Tudo bem B., vem sim

Eu não quero terminar por telefone ou mensagem

\textbf{B.} 31 de janeiro de 2014 às 11:50

Na boa, mudei de ideia, não vou praí não. Atende o telefone aí

\textbf{P.} 31 de janeiro de 2014 às 11:50

B. não dá, tô com o bebê no colo

Vamos nos falar de noite

\textbf{B.} 31 de janeiro de 2014 às 11:52

Não

\textbf{P.} 31 de janeiro de 2014 às 11:52

Uma já tá nervosa

\textbf{B.} 31 de janeiro de 2014 às 11:53

Quando você tiver uma resposta clara sobre isso a gente conversa. Eu
quero você, te amo mas tô decepcionado contigo. Você me abandonou

\textbf{P.} 31 de janeiro de 2014 às 11:55

Preciso cuidar da nossa cria. Desculpe, vou desligar porque a prioridade
é ela. À noite falamos.

\textbf{B.} 31 de janeiro de 2014 às 11:56

Anna cuida dela aí 11:56 mas 11:56 eu tô totalmente magoado contigo
11:57 você me faz de bobo 11:57 jamais faria isso contigo 11:57 vim pra
cá pra Rio das Voltas com você 11:57 cheio de esperança 11:57
determinação 11:57 mesmo na tristeza que a gente tá passando 11:57 Uma
precisa da gente 11:57 mais você tem a certeza que aí 11:58 longe de mim
11:58 vai ser melhor pra ela 11:58 você poderia estar aqui 11:58 com os
cuidados da minha tia aqui em casa 11:58 com quatro mil na conta pra
contratar qualquer pessoa pra te ajudar com a cria 11:58 e eu te
ajudando do jeito que dá 11:59 quanto aos esquemas de trabalho aqui
11:59 ridículo 11:59 você me dizendo que se tivesse que vender água de
coco venderia 11:59 não voltou pra masmorra da sua mãe? 12:00 tsc 12:00
piada 12:00 eu tô aqui num vazio danado 12:00 me sentindo um idiota
12:00 por te acreditado em você 12:00 que você era forte 12:00 que mesmo
na escuridão você não ia me abandonar 12:01 sou um babaca mesmo 12:01 eu
confiando em você 12:01 acreditando que essa porra vai passar 12:01 que
a gente vai vencer mais uma 12:01 e você me volta pros seus inimigos
12:01 inimigos 12:02 você vai criar Uma sozinha 12:02 quem vai te ajudar
de verdade? 12:02 ninguém 12:02 a Lúcia e o Mauro 12:02 eles vão te dar
uma força 12:02 vai fazer seu doutorado lindo ai 12:02 e vai deixar ela
com quem? 12:02 na creche? 12:02 com sua mãe? 12:02 você não querer
ficar comigo eu entendo 12:03 mas você tá fodendo a vida de Uma pensando
dessa forma tão emocional 12:03 ela precisa estar com quem gosta dela
12:03 ela gosta de mim 12:03 você tá me tirando o minha filha de perto
de mim 12:04 e eu tô rezando aqui pra não surgir ódio se isso acontecer
12:04 eu sei que ela não tava bem 12:04 eu falhei em alguns aspectos
12:04 você também 12:04 ai você foge do problema 12:04 vai fazer eu
acreditar que eu me casei com uma mulher fraca covarde 12:05 que só dá
pra viver se ficar tudo no esquema 12:05 VOLTA PRA CASA 12:05 EU QUERO
MINHA FILHA 12:06 NÃO QUERO TER QUE APARECER AI 12:08 você desistiu da
missão de viver uma vida a dois 12:08 já te disse o Uma tava mal 12:09
mas e porque NOSSA FAMILIA ESTAVA MAL 12:09 sofrendo ataques de tudo que
é lado 12:09 e você com essa sua cabeça cheia de merda 12:09 tá achando
que vai resolver a porra toda aí 12:09 na casa da sua mãe 12:10 vai
viver a sua vida de mulher adulta 12:11 você não tava dando a
assistência correta pra sua filha 12:11 nem eu 12:11 mas fugir? 12:11
arregar? 12:11 tô inconformado 12:11 você tá acabando com o sentimento
mais puro que eu tenho 12:11 o amor que sinto por você 12:15 em nenhum
momento arreguei contigo 12:15 pensei em terminar sim 12:15 mas eu
terminei? 12:15 terminei? 12:15 te deixei na mão? 12:16 te abandonei?
12:16 todo dia 12:16 não 12:16 quando você acordava já surtando 12:16 eu
parava e refletia 12:16 é só um momento ruim 12:17 EU NÃO ACREDITÔ QUE
VOCÊ TA SE SEPARANDO DE MIM POR CONTA DE QUE A GENTE TAVA BRIGANDO MUITO
AQUI 12:17 isso e totalmente ridículo 12:17 ridículo 12:18 tá a
prioridade é Uma 12:18 mas aí 12:18 você tá correndo do problema 12:18
correndo pra debaixo da mesa 12:19 igual criança com medinho 12:19 para
de arregar 12:20 porra até parece que nunca precisou lutar pela vida
12:24 eu tô tentando não ficar com raiva de você Anna 12:24 mas tá doído
ver que ver arregou 13:06 vou tentar dormir um pouco 13:06 tô muito mal
13:30 deixa pra lá 13:30 não liga mais pro que eu falei 13:31 tô com
medo e confuso 13:31 inseguro e magoado 13:31mas entenda 13:31 só queria
seu bem

\chapter{O tédio, o ódio e o nojo}

Meu subconsciente é tão arrogante que vive corrigindo o Drummond

\chapter{Uma fonte inesgotável}

\textbf{P.} Sábado, 1 de fevereiro de 2014 às 06:42

B.

\textbf{P.} Sábado, 1 de fevereiro de 2014 às 06:42

Eu tô arrasada por causa de ontem

Uma ficou nervosa, insegura, apavorada mesmo

Quase não dormiu

Eu perdi todo o trabalho que eu tava fazendo com tanta dificuldade pra
ela dormir

Novamente tô levantando da cama sem energia pra cuidar dela

E ela tá bem mal

Então, pensa na sua filha e tenta compreender

Eu vou desligar aqui

\textbf{B.} 1 de fevereiro de 2014 às 06:45

Não

\textbf{P.} Sábado, 1 de fevereiro de 2014 às 06:45

Não vou atender o telefone

\textbf{B.} 1 de fevereiro de 2014 às 06:46

Não Anna ´para com isso

para de me rejeitar

\textbf{P.} Sábado, 1 de fevereiro de 2014 às 06:46

Não vou falar com você hoje porque tenho, TENHO que cuidar do nosso bebê

\textbf{B.} 1 de fevereiro de 2014 às 06:46

Eu só quero seu bem

\textbf{P.} Sábado, 1 de fevereiro de 2014 às 06:46

A gente se fala amanhã

\textbf{B.} 1 de fevereiro de 2014 às 06:46

Eu não sou se problema

\textbf{P.} Sábado, 1 de fevereiro de 2014 às 06:46

Hoje não

Amanhã falamos

\textbf{Beltrano}, 1 de fevereiro de 2014 às 06:47 ANNA VOCÊ ME QUER
COMO UM INIMIGO MESMO EU SÓ TÔ QUERENDO FICAR BEM POR QUE VOCÊ NÃO
COLABORA EU NÃO TÔ COM A MININA VONTADE DE FAZER MAL PRA VOCÊS ENTÃO
PARA DE ME TRATAR COMO UM MONSTRO VOCÊ É MELHOR QUE ISSO DE ONDE VOCÊ
TIROU ESSA IDEIA DE QUE EU TÔ TE FODENDO QUE SEU TIVESSE CUIDADO DE VOCÊ
DIREITO NÃO PRESCISAVA VOCÊ TA AÍ COMO ANNA EU ME DEDICANDO PRA VOCÊ
AQUI VENDO VOCÊ FICAR DOENTE DO MEU LADO VOCÊ TÁ ME FERRANDO SE RELMENTE
UMA É PRIORIDADE POR QUE VOCÊ ME ATACA DESSA FORMA E DEPOIS FOGE NÃO ME
ENCARA DE FRENTE TÁ COMEÇANDO A ME DEIXAR COM RAIVA DE VOCÊ E ACHO QUE É
ISSO MESMO QUE VOCÊ QUER QUE EU FIQUE COM RAIVA ÓDIO FALE MERDA PERCA A
LINHA QUE AÍ FICA FÁCIL PRA VOCÊ CORRER DO BARCO VAI TER UM MOTIVO PRA
VOCÊ SE ENCANAR NOVAMENTE PARA COM ESSA BABAQUIÇE VOCÊ NÃO ME EXPLICA
COMO ESTÃO AS COISAS AÍ EU FICO SEM NOTÍCIAS VOCÊ SÓ ME ATACANDO E
DENTRO DA CASA DA PESSOA QUE MAIS TE FODEU ISSO NÃO TÁ SENDO RACIONAL OU
VOCÊ TÁ DE MUITA MAS MUITA MALDADE NÃO CONHEÇO GENTE RUIM DE VERDADE
VOCÊ TÁ EM OUTRA CLASSE JÁ PORQUE VOCÊ TEM QUE ME MATAR PORQUE VOCÊ NÃO
TEM A CORAGEM DIGNIDADE POSTURA DE MULHER DE ME ENCARAR DE FRENTE
CORRENDO DE MIM CARA EU NÃO SOU SEU INIMIGO VOCÊ DIZ QUE TÁ MAL QUE TÁ
SOZINHA COM A CRIA AÍ CADÊ A SUA MÃE CADÊ OS MILHÕES DE AMIGOS AÍ SEI
ACHO QUE QUEM TÁ NA SITUAÇAO DE TER COMIDINHA PRONTINHA SÓ CHEGAR E
COMER TEM UM BABÁ PRA CUIDAR DE VOCÊ E DA UMA E AINDA UMA FONTE
INISGOTÁVEL DE GRANA VOCÊ TÁ FALANDO MAIS DE VOCÊ O TEMPO TODO DO QUE DE
MIM MESMO ENTÃO ESSE ERA O DITADO QUE VOCÊ COSTUMAVA FALAR ALGUMA COISAS
DE PEDRO SEI LÁ

\chapter{Tá doendo}

— Tudo bem?

— Tudo indo.

\chapter{O que João diz de Pedro fala mais sobre João do que sobre Pedro}

SÓ SEI NA BOA EU POSSO TÁ MORRENDO DE DESESPERO AQUI MAS EU TÔ LIMPO SEI
QUE FIZ O MELHOR E VOCÊ É VOCÊ FODA-SE NÉ VOCÊ NÃO SENTE CULPA MESMO SE
ACHA NO DIREITÔ DE ME ENCANAR TIRAR MEU FILHA DE PERTO DE MIM DIZER
COISAS DESNECESSÁRIAS SABE QUAL É SEU PROBLEMA VOCÊ PRECISA TÁ SEMPRE
NUM LUGAR ONDE AS PESSOAS FALEM O QUE VOCÊ QUER ESCUTAR TÔMEM ATITUTES
SEMPRE BASEADAS EM VOCÊ VOCÊ NÃO SABE LIDAR COM A CULPA SEMPRE PASSA A
PETECA PRA FRENTE VOCÊ SE ENGNA O TEMPO TODO INVENTA COISAS PRA SUA
CABEÇA CRIA SEU MUNDO E SE ACHA ESPECIAL NA BOA PARA CARALHO DURANTE
ESSA SEMANAS QUE EU TÔ SOZINHO AQUI PUDE OBSERVAR MELHOR AS COISAS CARA
QUANTA MULHER POBRE COMENDO PÃO COM MANTEIGA PRA SOBREVIVER E MINHA
ESPOSA CHEIA DAS OPORTUNIDADES FAZENDO ISSO E MUITÔ TRISTE VER QUE O
PESSOAL QUE TRABALHA NA COLETA TEM VARIAS MULHERES SEM DENTE FUDIDA
ESSES DIAS UMA SENHORA ME PEDIU AGUA TODA SEM GRAÇA PÔ A GENTE TROCANDO
IDEIA AÍ COMEÇAMOS A FALAR DA VIDA UM CADIM COISA DE DOIS MIN E ELA
FALANDO QUE JA OUVIU UMA GRITANDO AÍ EU FALEI QUE UMA É FODA GRITA MESMO
DANDO RISADA E A MULHER ME VIRA E FALA NOSSA TENHO CINCO LÁ E EU NOSSA
QUANTO FILHO PÔ A MULHER LEVANTA SETE DA MANHA DÁ CONTA DE GERAL SAI PRO
BATENTE QUE É FICAR LIMPANDO GUIA VOLTA PRA CASA E TERMINA O ROCK E NA
MORAL ELA TÁ SOZINHA NÃO TEM MARIDO NÃO TEM MÃE ANNA DIANTE DAQUILO EU
SENTI VERGONHA VOCÊ ENTENDE VERGONHA VOCÊ TÁ DOENTE EU SEI MAIS PÔ VAMO
PARAR PRA PENSAR MELHOR COMO EU DISSE OLHA PRA DENTRO DE VOCÊ EU FICO O
TEMPO TODO ARRUMANDO SOLUÇÃO PRO SEUS PROBLEMAS E VOCÊ TAMBÉM ÉRAMOS UM
CASAL MAS VER VOCÊ FRAQUEJAR ASSIM TÁ FODA PARA COM ESSA PALAHAÇADA DE
QUE NÃO VOU FALAR CONTIGO TENHO QUE CUIDAR DE UMA VOCÊ TÁ É FUGINDO DO
PROBLEMA VOCÊ NÃO CONSEGUE ME ENCARAR DE FRENTE PORQUE NO FUNDO VOCÊ TÁ
FAZENDO FEIO PRA CARALHO TÁ FUGINDO DE MIM ARREGANDO TUDO BEM MAS NUNCA
PRECISOU VOCÊ FAZER ISSO ERA SÓ VIRAR E FALA B. TÁ RUIM AQUI E EU IA TE
AJUDAR COMO SEMPRE ENTRE TRANCOS E BARRANCOS AGORA VOCÊ TÁ AÍ FAZENDO
ESSA SACANAGEM TODA CHEIA DE SI AGINDO COMO SE FOSSE A GAROTINHA QUE
FICOU SEM O SORVETE NA BOA VOCÊ TINHA UM MARIDO AQUI UMA FAMILIA E TÁ
SENDO INFANTIL NO QUE VOCÊ TÁ FAZENDO VOCÊ FICAR FALANDO B. VOCÊ E
ADULTO AH ANNA VOCÊ TA TÔTALMENTE INFANTILIZADA TÁ AGINDO IGUAL
MENININHA QUE MORAVA AÍ COM A MAMÃE SAI DAÍ PORRA TÁ RUIM SAI DAÍ VOLTA
PRA CASA PRO SEU LAR NÃO EU TENHO QUE FICAR AQUI E POR UMA VAMO VER ATÉ
ONDE VAI ESSA FITA COMO DISSE VOCÊ PODE MUITÔ BEM FAZER O QUE QUISER DA
SUA VIDA VOCÊ TEM CONDIÇÕES PRA ISSO MAS SER COVARDE NÃO ISSO NÃO UMA
NÃO MERECE FICAR SEM UMA FAMILIA PORQUE A MÃE DELE TÁ AGINDO FEITO UMA
COVARDE EU TÔ SEM COMO ME DEFENDER DISSO TUDO AQUI VULNERÁVEL E MESMO
ASSIM CHEIO DE VONTADE DE QUE ISSO TERMINE BEM MAS VOCÊ TÁ SÓ VENENO E
ISSO É SEU PESSOAL VOCÊ ESTRAGA TUDO NA SUA VIDA COM ISSO DESDE MUITO
TEMPO ATRÁS PORRA VOCÊ VIROU PIADA PRA SUA FAMILIA PROS SEUS EX MARIDOS
ATÉ PRA ESTRANHOS QUE ISSO MULHER FORÇA AÍ VOU FUMAR UM CIGARRO LARGA DE
SER ARREGONA você ta saindo daqui porque você tá afim de ficar de boa AÍ
você não convence ninguém com essa história porra eu sei que você TÁ se
fodendo mas não É assim que resolve agindo feito uma louca você não vai
morrer só precisa se cuidar melhor você TÁ nessa neurose eu sei que a
barra TÁ pesada mas você fica se arrastando pela casa aqui e falando que
eu não te dou suporte É foda daqui a pouco sua mãe vai mandar você
pastaR daÍ com essa criança É só ela se encher de você que pra ela não é
difícil aí ela te manda não sei quantos reais e vai mandar você seguir
seu rumo...

\chapter{Manga}

Sob a fina flor da pele

A carne boa no gosto

Não tinha nem cabimento

Era fruto do desgosto

Parecia não ter fim

Ao cabo de toda carne

De cabo a rabo caroço

Deu cabo de todo gosto

Acabou dizendo o azedo

Pareceu mangar de mim

\chapter{Não tenho dinheiro}

\textbf{P.} 12 de março de 2014 às 12:29

Taí? Preciso falar com você

Uma tá no peito ainda. Quer falar por aqui?

\textbf{B.} 12 de março de 2014 às 15:26

Só tava pensando nela e queria saber dela

\textbf{P.} 12 de março de 2014 às 15:28

Ela tá muito bem hoje. Dormiu até às 8h30 da manhã (inédito), depois
dormiu das onze até a uma a agora tá mamando e dormindo de novo

Você vai querer vir ver ela?

\textbf{B.} 12 de março de 2014 às 15:30

Não dá pra fazer isso agora

Eu não tenho dinheiro pra isso

\textbf{P.} 12 de março de 2014 às 15:33

Podemos tentar resolver isso da melhor forma

Não é favor pra você, faço pela nossa filha

Eu tenho dinheiro para as passagens e posso te receber aqui, só minha
mãe melhorar um pouquinho

Ou você pode ficar na Lúcia, se preferir

Meu apartamento tá quase alugado e talvez eu alugue um pra mim em breve

\chapter{Amor e vício}

Para ele eu dei tudo

Casa, comida, dinheiro, muita submissão, minhas convicções, a
oportunidade de estudar, uma cria.

Ele enrolou e fumou.

\chapter{Pai}

Para mim era uma espécie de pedra no sapato, um elemento opressivo que
minha mãe dizia que eu tinha que educar.

Durante a infância ele repetia periodicamente: Você não é bonita, você é
charmosa. Seu irmão é que é bonito.

Depois de adulta, o assunto era sobre pais que estupravam filhas.

Hoje ele insiste em histórias de idosos que foram abandonados para
envelhecer e morrer sozinhos.

\chapter{Sentido de autopreservação}

Eu não tinha nenhum

\pagebreak
\chapter*{}

\section{Fáscia}

Alma do músculo

\section{Alma}

Fáscia do ente

\section{Músculo}

Alma doente

\section{Ente}

Lama sendo

\chapter{Felicidade é não doer}

O mundo me dói o tempo todo. Café, álcool, espírito natalino,
oportunidades, tudo me descompensa. Detesto tomar remédio, mas vivo em
eterna hipocondria. É raro quando acordo e nada dói.

\chapter{Lirismo contemporâneo}

Doutor, a angústia voltou.

Dobramos o Lyrica?

\chapter{Voltei para a minha classe com o rabo entre as pernas}

Voltei com o rabo entre as pernas e uma criança de cinco meses no colo.

Voltei para o bairro onde eu nasci e voltei para a universidade.

Voltei para a família e para as amigas de antes.

Onde eu estava com a cabeça quando acreditei que o amor poderia saltar
sobre o abismo de classes brasileiro?


  \part{Uma criança de um ano}
%Parte superior do formulário

\chapter{Sobre o luto}

Não tive tempo

Ela só tinha seis meses

\chapter{Sobre o amor}

— Mãe, como foi que você casou com o meu pai?

— Ele foi a única pessoa que teve coragem de me esconder na casa dele na
época da ditadura.

— Pai, como foi que você casou com a minha mãe?

— Um dia ela foi jantar na minha casa e nunca mais saiu.

\chapter{Sobre o amor \textsc{ii}}

Dois aleijados querendo um completo

Água batendo no teto

\chapter{Orides}

O amor

Capenga de quatro pernas

E única mão

Esse amor

Não

\chapter{O que não nos mata nos fortalece}

Mas eu preferia ser mais bailarina e menos halterofilista

\chapter{Ansiedade}

Minha mãe sempre dizia

Que estava muito nervosa

Eu achava muito chique

Estar aquela palavra\medskip 

Que eu não sabia o que era

Mas que parecia adulta

Hoje eu tenho saudade

De não saber como era

\chapter{Rivotril}

É horrível

Mas sou sensível

\chapter{24/\,7}

Preciso pagar o café

Preciso pegar o livro

Preciso passar na farmácia

Preciso estudar para a prova

Preciso fazer render

Preciso fazer feijão

Preciso terminar hoje

Amanhã é feriado

Preciso deixar a casa arrumada

Ela chega muito cansada

Preciso estar bem quando ela chegar da creche

Preciso melhorar logo dessa dor

Preciso dormir para poder completar meus sonhos.

\chapter{O amor como sonhar pelo outro}

``Tive um sonho tão bonito com você! Era um fim de tarde de um dia
quente. Eu fui para uma estreia de um projeto de dança seu. Era quase um
solo. Você estava gravidíssima, quase nove meses! E o Vladmir Herzog era
codiretor da peça. Tinha um curta-metragem interagindo com você em cena.
Um curta-metragem em preto e branco contando o dia do casamento de um
casal na neve e você dançando tão, tão poderosa e ao mesmo tempo tão
sozinha.

Em uma parte, o palco foi tomado por homens muito diferentes em tipo de
corpo e altura, mas todos com a mesma roupa, mesmo cabelo, mesma barba e
chapéu. E eles ocuparam o palco de forma homogênea, como uma plantação
de eucaliptos que se move, mas se move muito lentamente. Então o vídeo
congela, essas pessoas todas em cena com você, eles quase imóveis e você
fluindo no meio deles com uma certeza e uma agilidade linda, enquanto a
voz do Herzog cantava uma música de ninar.

Acordei chorando de tão bonito que foi.''

\chapter{Outro amor}

Coloquei a criatura no peito, embrulhei no meu casaco, apertei bem e
disse "Pronto, filha, pode nanar. Você tá que nem na barriga. Você quer
voltar pra barriga da mamãe?"

A pequena cospe o peito, olha bem na minha cara, sorri e diz "Qué!"

\chapter{Não moooorde!}

Ai, que linda, dando beijo na mão da mamãe! Só não morde, tá, filha? Não
morde! NÃO MORDE!

Explode coração!

Ademilde Fonseca canta "sossega leão, sossega leão"

A pequena dança, sorri e manda um "wraaaaawwww"

\chapter{A caminho do amor}

A brincadeira é ir até a cadeira, soltar as mãos e correr rindo e
gritando para dar um abraço na mamãe.

Questões de gênero

— Tó filha

— Bigadu

— De nada

— Tó mamãe

— Obrigada

— De nado

\chapter{O gênero do pé}

Tem sapato pra bebê que tá começando a andar?

— É pra menino ou menina?

— É pra usar no pé.

Nana, Nenê

"Nana neneeeee ohvqçwoehe\textasciitilde{}rgjwpjd\textasciitilde{}efj
ggaaaaaa

Papai k.rhgfjqwpodjoefhhfwo çaaaaaa

Mamãe lehhçfdlflbjlf aaaaaaaaaaaaaaaaaaaaa"

\chapter{Fonética}

— Uma, essa é a cartola.

— Calalha!

— Não, filha, Car-to-la.

— Ca-la-lha.

— E essa é a borboleta.

— Beceta.

— Borboleta.

— Beceta!

— Bor-bo-le-ta.

— Be-ce-ta!

\chapter{Boinha di tabão}

— A mamãe gosta da Uma.

— A mamãe gota da Uma?

— Gooosta. E a Uma, gosta de quem?

\chapter{Filha de ateia}

— Nossa senhora, Uma, quanto cocô!

— Nota tinhora! Quanto cocô!

— Quanto cocô, nossa senhora!

— Tinhoraaaa, cadê bucêeee, eu bim aqui só pra ti bêeeeee!

\chapter{Razão}

— Fiquei brava com você, mamãe.

— Ah, filha, eu não te tiro a razão...

— Não?!

— Não.

— Tá bom, mamanhê.

\chapter{Conversa de mães}

E a gente achava que fazer doutorado era difícil...


  \part{Uma criança de dois anos}

\chapter{Quero ser grande}

— Ih, filha. Mamãe já tá precisando comprar outro chinelo pra você.

— Ú quéio quiechê, mamãe.

— Quer crescer pra que, Uma?

— Pá compá sozinha.

\chapter{Tricerátopo}

— Ú qué ficá gandhi.

— Quer ficar grande pra que, filha?

— Pá falá ticerápt.

\chapter{Gramática gerativa}

— Brincou bastante, filha?

— Brinquei.

— Você gostou?

— Gostei.

— Já comeu?

— Comei.

\chapter{Grandezas}

— Eu sou grande!

— Você é grande, filha?

— Enorme!

— E desde quando?

— Cinco horas.

\chapter{Sobre o tempo}

— Olha, uma galinha! Igual à que a gente viu.

— É verdade, filha. Você lembra quando a gente viu?

— Eu lembro.

— E quando foi?

— Semana que vem.

\chapter{Bailarinas}

— Mamaaaãe, olha! Eu tô dançando! Igual bailarina!

— Que legal, Uma! Você gosta de dançar?

— Eu gosto!

— A mamãe também gosta. Sabia que eu era bailarina?

-É?... Quando eu era velhinha eu também era bailarina.

\chapter{Grandezas \textsc{ii}}

— Ai, filha, eu gosto tanto de você!

— Gosta?

— Muito!

— Eu também gosto de você, mamãe.

— Gosta? Quanto?

— Quarento!

\chapter{Topoi}

— Uma, a mamãe não tá achando o chinelo dela. Você sabe onde tá meu
chinelo?

— Eu sei, mamãe.

— Onde?

— Tá em algum lugar.

\chapter{Castígolo}

— Você é chata! Você tá de castígolo!

\chapter{Por causa de você bate em meu peito...}

— Tá contente, filha?

— Eu tô.

— Por quê?

— Porque eu tô feliz.

— Tá feliz por quê?

— Por causa de vochê.

\chapter{Uma boa pergunta}

— Eu quero o carrinho!

— Tá, mas vamos trocar essa fralda primeiro?

— É uma boa pergunta...

\chapter{No banho}

— Uma, precisa lavar a cabeça.

— Não, não precisa! Eu não gosto dessa brincadeira!

\chapter{Três barrigas}

— Mais iogurte, filha? Não sei como cabe tanto iogurte na sua barriga.

— Eu tenho três barrigas.

\chapter{Pernilongo}

— Volta aqui, seu pernilongo canalha!

— Não é canalha, mamãe, é pernilongo!

\chapter{Pernilongo \textsc{ii}}

— Ihhh, filha, o pernilongo picou a mamãe...

— Ohhhh... tudo bem... vai passar...

\chapter{Pernilongo \textsc{iii}}

— Maaaãe, pernilooongo!

— Pronto, matei.

— Eu vou matar de novo!

\chapter{Agora eu não posso?}

— Mamaaaãe, um mosquito! Dá a raquete!

— Deixa que a mamãe usa a raquete.

— Mas eu achei o mosquito! Agora eu não poooosso?

\chapter{Mundo literal}

— Uma, sabia que a mamãe morre de amor por você?

— Morre?

— De amor por você.

— Já morreu?

\chapter{Mundo literal \textsc{ii}}

— Que bom, né filha? A gente foi no show, mamãe encontrou os amigos, as
amigas, trocou telefone...

— E onde tá nosso telefone agora?!

\chapter{Mundo literal \textsc{iii}}

— Quero mamadeira, mamãe.

— Vou pegar, filha. Mas fica na cama, vai chamando o soninho.

— Soninho... soniiiiiinho... soninhooooo...

\chapter{Sim ou não?}

— Quer laranja, filha?

—

— Quer, Uma?

—

— Responde pra mamãe, sim ou não?

— Sim ou não.

\chapter{Persuasão}

— Mamãe, quero ir pra pracinha!

— Ah, filha, tem certeza que não quer tirar uma soneca antes?

— Tenho certeza.

— A mamãe queria descansar um pouquinho...

— Mas não pode. Vem mamãe, você vai adorar a pracinha!

\chapter{Colossal}

— Filha, dá a mão pra mamãe que essa escada é muito grande.

— É colossal?

\chapter{Voz macia}

— Quer deitar na minha almofada, mamãe? Ela é macia...

Não pode batê

— Nossa, filha, a gente tá cansada, né? Acho que a gente vai bater na
cama e dormir.

\chapter{Muito}

— Então vamos comer bolo lá na casa da vizinha, vou me vestir. Você já
tá vestida, filha?

— Já. E tô muito bonita!

\chapter{Verdade}

— Você vai pra escola e mamãe vai trabalhar.

— De quê?

— Mamãe é escritora.

— Não é não.

— E qual o trabalho da mamãe? Cuidar da Uma?

— É, cuidar da Uma.

— Verdade. Mas quando você vai pra escola, o trabalho da mamãe é ler e
escrever.

— Aaahh, isso não parece muito bom...

— Por que, filha?

— Porque é muito chato!

você consegue muitas coisas

— Ih, filha, não tô conseguindo fazer esse fantoche de meia, tô ficando
chateada.

— Ah, você consegue muitas coisas, não fica chateada!

\chapter{Sujado}

Uma pega a caneta, rabisca a parede toda e depois corre pra mim:

— Limpa aqui, mamãe, tá sujado! A parede tá sujado!

\chapter{A mamãe!}

— Filha, quem faz a bagunça é que arruma. Uma bagunçou, quem tem que
arrumar?

\chapter{Bagunça}

(Assistindo desenho)

— Por que ela tá brava?

— Ela tá brava porque todo mundo fez bagunça e só ela tá arrumando.

— É?

— É. Não pode. Todo mundo bagunçou, todo mundo arruma.

— Outro dia eu cheguei o dinossauro fez bagunça.

— É mesmo filha? E ele não quis arrumar?

— É. Eu fiquei brava. Não pode!

\chapter{Bagunça \textsc{ii}}

— Mamãaaaae, macarrãaaao!

— Tô fazendo, filha. E você, o que tá fazendo?

— Bagunça.

\chapter{Desmatamento}

— Mamãe! Você matou meu dinossauro! Eu vou desmatar!

\chapter{Limites}

— Fica aí tiranossauro, não pode voar lá no céu. Você não é pterodátilo!

\chapter{Direitinho}

— Qual é o nome desse?

— Diplódoco.

— Diplódoco! Isso mesmo, mamãe! Você acertou direitinho!

\chapter{Erradinho}

(Uma) Esse é o bebê do estegossauro. E esse é o bebê do Rex. E esse é o
bebê do...

(Eu) Parassaurolofo!

(Uma) Não! Esse é o bebê do paquicefalossauro.

\chapter{Giganotossauro}

— Giganotossauro? O que será que é Giganotossauro, filha?

— Não sei... é uma coisa... fantástica...

\chapter{Ocupada}

— Uma, ficaram dois dinossauros na sua banheira.

— Pega, mamãe.

— Ué, você não pode pegar?

— Não, tô ocupada.

\chapter{Maior que grande}

— Ele tá de boca fechada?

— O estegossauro tem uma boquinha bem pequenininha porque ele só come
folhas.

— E o Rex?

— O Rex tem essa bocona cheia de dentes pra comer todos os bichos que
ele quiser.

— Nooooossa... que maioooooor!

\chapter{Dáctilo}

— O que você tem na boca, Uma, só o dedo?

— Só. Tá gostoso.

— Tem gosto de quê?

— Dinossauro!

\chapter{Raízes}

— O tricerátopo tem chifres porque ele come raízes.

— Éeee? Num sabia...

— E as raízes ficam embaixo da terra. A gente também come raízes.

— Eu não como!

— Come sim, Uma. Você adora batata doce, batata, cenoura, beterraba...

— Não, eu não gosto! Eu não vou comer nunca mais!

\chapter{Febre}

— Uma tá bebê.

— Tá bebê, filha?

— Tá.

— Tudo bem, filha. Quando a gente fica doente fica meio bebezinho mesmo.

-É? Num sabia...

\chapter{Flores}

— Bom dia, flor do dia.

— Bom dia, mamãe flor do dia.

\chapter{Paquicefalossauro}

— Ai, Uma, não é pra me dar cabeçada!

— Eu dou cabeçada sim, eu sou um paquicefalossauro!

\chapter{Verdade ii}

— Quem te ligou na tomada, criatura?!

— A mamãe!

— É verdade...

\chapter{Tá escrito}

— Vamos tomar café na padaria?

— Vamos! Eu quero brigadeiro!

— Brigadeiro de manhã não pode, lembra que a gente combinou?

— Lembro. Tá escrito.

\chapter{Tá escrito ii}

— Mamãe, o que você tá fazendo?

— Eu tô lendo o meu livro.

— É de dinossauros?

— Não, filha, esse livro não tem dinossauros, só tem letras.

— Tá escrito: não-tem-di-no-sau-ro-não.

\chapter{Demais}

— Boa noite, filhinha, te amo demais.

— Te amo demais também, viu mamãe?

\chapter{Gerúndio + infinitivo}

— Uma, vamos tomar banho?

— Pera aí, tô terminando de acabar.

\chapter{Porque eu gosto}

— Uma, a gente fez tudo que você queria, por que você tá emburrada?

— Porque eu gosto de emburrar.

\chapter{Vai e vem}

— A Isabel foi embora!

— É, filha, mas ela volta.

— Eu não quero que ela volta, eu quero que ela vem!

\chapter{Crescer}

— Precisa comer pra crescer, não é mamãe?

— É, filha, pra crescer, pra ficar forte...

— Pra carregar coisas pesadas... Quando você ficar pequenininha, eu
carrego pra você!

\chapter{Sonho}

Uma desperta da soneca, ``O pavão voa, voa... Como o pteranodonte. Ele
gosta de voar...'' e volta a dormir.

\chapter{Sobre o tempo}

— O que o braquiossauro come, mamãe?

— Folhinhas.

— E o rex, mamãe?

— O rex come carne.

— E o pteranodonte?

— Peixes. E a gente come todas as coisas: raízes, folhinhas, carne,
peixe...

— Baleias... Eu gosto muito de comer baleias!

— E quando foi que você comeu baleia, filha?

— É porqueee... eu gosto muito de comer baleias!

— E quando foi que você comeu?

— É porqueee... eu como muitas baleias! Porque eu gosto!

— Onde você comeu baleia, Uma?

— Na escola!

\chapter{Predadores}

— Mamãe vai pendurar roupa.

— Quero também!

— Você pode me ajudar, pode me passar os...

— Pregadores! Igual dinossauro!

\chapter{Engano}

— Quem era, mamãe?

— Era engano. Uma pessoa ligou aqui procurando uma tal de Cláudia, aqui
não é a casa da Cláudia, então foi engano.

— Alô, Cláudia?

\chapter{Toma uma pastilha}

— Não quero, é muito ardida! É encardida!

\chapter{Obliquidades}

— Uma, vou pegar as sacolas, você me ajuda?

— Eu me ajudo!

\chapter{Filha de ateia \textsc{ii}}

(Passando pela igreja de nossa senhora de sei lá eu)

— Olha, mamãe, que castelo grande!

\chapter{Filha de ateia \textsc{iii}}

— Tchau, Uma, vai com deus!

— Se escondeu?

\chapter{Um, dois e muitos}

— O estiracossauro tinha muitos chifres, né?

— Muitos! Milhões e milhões! Três!

\chapter{Adjetivos}

— Mamãe, você é boa!

— Você também é uma criança muito boa, filha.

— Você é legal, você é liiinda!

\chapter{Ao despertar da soneca}

— Eu tenho um olfato ótimo, igual o rex, pra cheirar carniça.

\chapter{Sábado}

— Qual o seu nome?

— Uma.

— Você está na escolinha, Uma?

— Não, hoje é sábado.

\chapter{Gramática gerativa \textsc{ii}}

— O tricerátopo tem chifres pra chifrar os predadores. E o estegossauro
tem espinhos na cauda pra espinhar.

\chapter{Gramática gerativa \textsc{iii}}

(Desligando o interruptor)

— Escurei!

\chapter{No meio da noite}

— Mamãe! Acende a luz! Me dá colo! Você tava escurecida...

\chapter{Explodindo o dino}

— Eu vou explodar o dino! Explodei!

\chapter{Brincando com o tiranossauro}

— Ele é o jurássico e eu sou o cretáceo.

\chapter{Maternidade e pós graduação}

— Eu vou entregar o trabalho antes porque não posso fazer nas férias. Se
você achar que ficou muito ruim dá tempo de melhorar.

— Eu não sei qual é o seu parâmetro, mas acho difícil seu trabalho ficar
muito ruim.

— Meu parâmetro é alto, forte... meu parâmetro é halterofilista! Eu
odeio o meu parâmetro!

 

\chapter{No meio da noite \textsc{ii}}

— Mamãe, me dá a mamadeira por favor? Obrigada.

\chapter{De manhã}

Uma se enrosca em mim e diz baixinho:

— Sua maravilhosa...

\chapter{De manhã \textsc{ii}}

— Mamãe, eu não quero ir pra escola! Eu quero ficar aqui com você.

— Mas eu não vou ficar aqui, eu vou trabalhar.

— Por que você precisa trabalhar?

— Pra poder comprar as coisas, roupa, comida...

— Dinossauros... Mamãe, quero ir pra escola!

\chapter{Cantando}

``Um meteoooro

Caiu sobre a Teeerra

E bloqueou a luz do Soool

Ou foi um vulcãaao

E os dinossauros moreeeram''

\chapter{Cardápio}

— Uma, vou fazer almoço, o que você quer comer?

— Tricerátopo misturado com arroz.

\chapter{Coprólitos}

— Uma, vem guardar a massinha.

— Não.

— Mas ela vai ficar dura e não vai dar mais pra brincar.

— Vai virar um cropólito!

\chapter{Lição de coisas}

— Uma, desce da mesa! Já falei mil vezes, você pode cair!

— Eu sei disso.

— Então por que continua subindo?

— Porqueee... eu sou criança!

\chapter{Psicologia cretácea}

— Mamãe, eu não quero tomar banho!

— Não, filha? E você vai dormir na casa do apatossauro? Lá todo mundo é
fedido...

— Não! Eu vou tomar banho! Mas eu não quero escovar dente!

— E você vai dormir na casa do T-Rex? Lá todo mundo tem bafo de
carniça...

\chapter{Muito pouquinho}

— Uma, vamos embora que você já tá com muito sono!

— Eu não tô com sono!

— Nem um pouquinho?

— Eu tô um pouquinho com muito sono...

\chapter{Cuspitossauro}

— Uma, que novidade é essa de ficar cuspindo? Parece uma hiena...

— Eu sou o cuspitossauro!

\chapter{Água na via láctea}

— Uma, não joga a água da banheira fora, a gente vai usar essa água.

— Por quê?

— Porque senão é desperdício e acaba a água do planeta.

— Hahaha, hahaha, hahaha...

— Que foi, filha?

— Aqui não é planeta, aqui é via láctea!

\chapter{Trava língua}

— Uma, você já perdeu um tricerátopo e um paquicefalossauro fazendo
isso! Se você perder o parassaurolofo eu não vou te dar mais dinossauro
nenhum!

\chapter{Gramática gerativa \textsc{iv}}

— Sua abastada!

— O que é abastada, filha?

— É uma coisa que abasta.

\chapter{E agora?}

— Mamãe, qual a diferença entre o euplocéfalo e o ancilossauro?

\chapter{Empatia}

— Ih, filha, começou a chover! Bem na hora que a gente ia pra praia! Que
droga, não vai dar pra ir.

— É, que droga, não é?

— É, que droga mesmo. Mas vamos aproveitar pra almoçar que eu tô
morrendo de fome.

— Que bom que choveu, não é mamãe?

— Por que, Uma?

— Porque você tava com fome.

\chapter{Adormecer}

— Eu vou ficar olhando pra você, tá mamãe? Porque você é muito
bonitinha.

\chapter{Dia dos pais}

Uma já parou de acordar antes das seis. Eu ainda não.

\chapter{Genealogia}

Pai de uma criança, pai ausente, pai biológico, progenitor, genitor,
sombra genética, aquele que está longe, desfibrado, desamado, nada,
ninguém, puro cansaço.

\chapter{Dorme, mamãe}

— A mamãe...

— Filhinha...

— A mamãe...

— Sim, filha, eu sou sua mamãe.

— E eu sou o papai?

— Não, filha, seu papai chama Beltrano. Você tem um papai, mas ele tá
longe.

— É? Eu tenho um papai?

— Tem, filha. Seu papai é o Beltrano.

— E ele tá longe?

— Ele tá longe, filha. Você que ver fotos dele?

— Quero!

— Olha, seu papai é esse aqui. O Beltrano. Ele é bonito...

— (Rindo) É engraçado!

— (Chorando) É engraçado, filha?

— É engraçado.

— Esse é o seu papai.

— Tá bom, mamãe.

— (Chorando) Tá bom?

— Tá.

— (Chorando)

— Dorme, mamãe.

\chapter{Beijinho}

— Olha, ela não parece muito bem.

— Quem não parece muito bem, filha?

— A Uma.

— Quer um beijinho pra sarar?

— Não, não adianta.

  \part{Uma criança de três anos}

\chapter{Falta de alternativa}

— Nossa, mas você cria ela completamente sozinha?! Como você consegue?

\chapter{Benesse}

— Hoje quando você voltar da escola vai ter arroz e feijão fresquinho,
franguinho com batatas, tomatinho e salada de alface.

— E você vai fazer tudo isso sozinha, sem ajuda?

— É, filha.

— Mas se você precisar, você me chama que eu venho ajudar você, tá
mamãe?

\chapter{Comendo}

— Uma, qual é a diferença entre almoço e janta?

— É que os dois são iguais.

— Deixa eu te explicar: quando a gente acabou de acordar, a gente toma
café da manhã; quando está de dia, a gente almoça e quando está de noite
a gente janta. Então, o que você está fazendo agora?

\chapter{Ninando}

— "Quem sabe / o super homem venha nos restituir a glória/ mudando como
um deus o curso da história/ por causa da mulheeeer"

— Da mulher maravilha!

\chapter{Direitos}

— Eu quero ver desenho...

— Não, filha, vamos levantar, trocar de roupa, a gente toma café na
padaria...

— Não! Desenho! Eu quero desenho! Todo mundo devia ter o direito de ver
um desenho na hora que acorda!

\chapter{Presente}

— Olha, mamãe, eu fiz um presente pra você!

— Você fez um desenho!

— É, e eu escritei também!

\chapter{Trabalho}

(Colocando pilhas no brinquedo novo)

— Uma, me passa a chave de fenda? Obrigada.

— De nada.

— Agora me passa a pilha? Obrigada.

— De nada, só estou fazendo o meu trabalho.

\chapter{Só um pouquinho}

— Pôxa, Uma, mas já quebrou?! Um brinquedo tão lindo...

— Agora ficou só um pouquinho lindo, não é mamãe?

\chapter{Nessa cidade}

— Uma, por que você está gritando desse jeito?

— É que eu sou um parassaurolofo.

— Só se for um parassauroloko, gritando assim sem necessidade.

— Tem cidade sim! A cidade dos dinossauros!

\chapter{Um bicho animal}

— Os dinossauros não são bichos, eles são animais!

— É, filha? Qual a diferença entre bicho e animal?

— É que os animais são mais fortes e os bichos são mais inteligentes.

— E o trodonte, que é um dinossauro inteligente?

— Ele é um dinossauro bicho.

\chapter{Meu bem querer}

— Filha, você vai brincar na casa do vizinho mas eu não vou ficar, tá?

— Tá, mamãe, descansa um pouquinho enquanto a gente brinca.

\chapter{Dizemil}

— Filha, quantos pães de queijo você vai querer?

— Três, quatro, cinco, seis, dezesseis, dezedoze, dezecatorze, dezeum,
dezetreze, dezequatro, dezemuitos!

\chapter{Grávida}

— Caiu uma árvore na frente da escola e cobriu dois carros, isso é muito
grávido, não é mamãe?

\chapter{Íngride}

— Mamãe, me ajuda, eu não consigo descer! É muito...

\chapter{José campera!}

— Olha, mamãe, que cachorro fofinho. Fofinho é um bom nome pra ele.

— E aquele outro ali, Uma, que magrinho. Ele pode chamar magrinho.

— É, magrinho!

— E aquele pequenininho ali?

\chapter{Lógica}

— Uma, vamos embora que tá chovendo!

— Não! Vamos chamar o fogo pra apagar a água!

\chapter{Não mesminho}

— Filha, você não vai comer nada?

\chapter{Como explicar a homossexualidade para seus filhos}

— Uma, você gostou dos meus amigos?

— Gostei.

— Eles são namorados.

— Eu sei.

— É um casal bonito, né?

— É, muito bonito.

\chapter{Gênero jurássico}

"Tem dinossauro fêmea, tem dinossauro fêmeo..."

\chapter{Empatia \textsc{ii}}

— Mas como eu saí da sua barriga se não foi pela pepeca?

— A médica cortou a minha barriga e tirou você lá de dentro.

— E depois?

— Depois ela costurou tudo bem direitinho e ficou tudo bem.

— Que bom, mamãe, fico feliz que você tenha ficado bem.

\chapter{Sete horas da manhã}

— Uma, eu acho o carcharodontossauro muito esquisito, com aquela cabeça
abaixada que parece que tá sempre cheirando o chão...

— É. Eita porra!

\chapter{Sete e vinte}

— Olha, Uma, aqui tá dizendo que o impacto do asteróide levantou rochas
que estavam a mais de vinte quilômetros de profundidade!

— É muito tempo, né?

\chapter{Extinção meteórica}

— Vamos brincar de dinossauro? Eu sou o velociráptor e você é o t-rex.
Corre que tá vindo uma extinção!

\chapter{Prolongamento vocálico}

— Mamãe, vou te contar uma história de terror! Era de noooite, tudo
escuuuro, uma floresta cheia de mooonstros e árvores de dentes
afiooosos...

\chapter{Quantidades}

— Eu te amo tanto, filha, você nem imagina!

— Imagino sim, dizemil horas!

Metro é quando algum pesa bem maior!

— Filha, o que é metro?

\chapter{Muito}

— Eu bebi três quilômetros de água! Eu tava com muita sede!

\chapter{Hipóteses}

— Poxa, acabei de limpar meus óculos e já estão sujos!

— É, eu não sei por que, né?

— Também não sei...

— Deve ser porque... é igual minha unha!

\chapter{Hipóteses \textsc{ii}}

— O infinito é uma bola bem grande que nunca para de crescer, né?

— Vou anotar isso que você falou do infinito, foi bonito.

— Haha! Você rimou! "Do infinito foi bonito", haha!

\chapter{Elegia}

— Ai, Uma, vamos tomar banho e ver um filminho...

— Não, vamos pra pracinha, por favor! Eu ainda tenho que gastar minha
elegia!

Como explicar a transexualidade para seus filhos

— Tá vendo essa moça, Uma? Ela nasceu com pipi, mas ela se sentia
menina, então ela virou menina.

— É?

— É, filha. Algumas pessoas nascem com pipi, mas se sentem meninas,
então elas são meninas; outras nascem com pepeca, mas se sentem meninos,
então eles são meninos. E algumas não se sentem nenhum dos dois.

— Eu tenho pepeca e me sinto menina.

— Então você é menina.

— É, cada um é o que é e o que precisa ser, né?

— Isso mesmo, filha!

\chapter{Amor/\,humor}

— Tatuuu...

— Quêee?

— Tá tudo bem?

— Tudo bem, tatu, e tu?

— Tudo bem também.

\chapter{De nunca!}

— Uma, para de bagunçar as coisas!

— É porque eu tô brava!

— Mas não adianta você ficar brava comigo, filha!

— É porque... Eu não gosto de você!

— Até parece... De onde você tirou essa ideia?

\chapter{Amor/\,humor \textsc{ii}}

— Sai do meu lugar, sua folgadinha!

— Ah, haha, eu não sou folgadinha!

— Folgada pacaramba!

— Ah, haha, eu não sou pacaramba folgada!

— Não, é folgada pacaramba, não pacaramba folgada, folgada pacaramba!

— Ah, haha!

\chapter{Amor/\,humor \textsc{iii}}

— Você é minha filha preferida.

— Mas eu sou a única!

— Ainda bem. Senão a outra corria o risco de perceber a minha
preferência enorme por você.

\chapter{Acordando}

— Olha, mamãe, o dia tá nublado.

— Acho que não, filhinha, vai fazer um sol lindo.

— Mas eu tô vendo o blado...

\chapter{Janela}

— Mãe, por que às vezes o dia não está chovente?

olhando os passarinhos pela janela

— Vamos, filha?

— Pera aí, tô vendo a poesia.

\chapter{Alta ficção}

Li para ela uma história onde a cabra briga com o Lobo a noite inteira.

— Mas ela ficou toda machucada de verdade ou de brincadeira, mamãe?

— De verdade na história, mas a história é uma brincadeira.

— Há! Brincadeira! Eu sabia disso!

\chapter{Zebra}

— Mãe, olha essa concha, ela tem barulho do mar! E olha essa, listrada,
acho que tem barulho de zebra!

\chapter{Filha de ateia \textsc{iii}}

— Mamãe, você é deus!

— Deus, filha, por quê?

— Porque eu tô brincando.

— E o que é deus?

— É o menino! Do céu! Ele sabe voar e come tudo que vê pela frente!

\chapter{Lindeza}

Enquanto eu arrumo o cabelo:

— Você tá linda!

— Obrigada, filha!

— Muito linda, muitíssimo! Você é uma lindeza!

Na hora de escovar os dentes

— Mamãaaae, eu tô com doi na péeeina! Eu tô com taaanta doi na péeeina
que não consigo nem ficar de péee!

— Nossa, filha! Melhor a gente ir pro hospital pra ver se tem uma perna
nova pra você que a sua tá podi, deve tá bichada!

\chapter{Você é bonita!}

Às vezes tomo banho torcendo para que ela adormeça enquanto me espera.
Mas ela grita do quarto:

— Mamãe!

— Quê, filha?

\chapter{Sol si poi}

— Mamãe, eles falaram "sol si pô", mas não é "sol si pô" é "sol si poi"!

\chapter{Odisseu}

— Boa noite, amormeuzinho.

— Eu não sou melzinho!

— Não, você é meu amorzinho.

— Eu não sou seu amorzinho!

— Não? Então quem é o meu amorzinho?

— Ninguém.

— Tá bom. Boa noite, Ninguém.

\chapter{Pestanas}

Imersa na contemplação da criança enquanto aguardo que ela adormeça...

— Mãe, de tanto você não piscar o olho você vai ficar sem pestana!

\chapter{Do amoi}

Pela manhã, como de costume, Uma me cobre de abraços e beijos:

— Ai, filhinha, eu gosto tanto de você! Sabe o que eu mais gosto de
você?

— Eu sei! Do amoi!

\chapter{Demonstrações esponâneas de amoi}

Rir das minhas bobeiras, embarcar empolgada nas cantorias, dançar junto,
me chamar de fofinha fofa, brincar junto com as palavras, prestar muita
atenção às minhas histórias, "dar carinho", me trazer uma ``foi''.

\chapter{Do amoi \textsc{ii}}

— Me abraça forte? Mais forte que uma pisada de alossauro!

\chapter{Dois maus}

Uma enrola um mouse na minha perna:

— Eu vou ver se você tá bem.

— Tô bem?

— Não, você tá mal.

— Tô mal, filha? O que eu tenho?

— Humm, alguma oxicidina...

— Nossa, e o que é isso?

— É tipo uma gripe. Você tá dois maus!

\chapter{Destrabalho}

— O dinossauro tá cansado de tanto destrabalhar.

— O que é destrabalhar, filha?

— É quando alguém destrói aquilo que o outro fez.

\chapter{Eu também}

— Mamãe, eu queria entrar dentro do desenho!

— Lá a gente pode fazer tudo que a gente quer, né filha?

— É...

— Mas assistir o desenho é uma forma de entrar dentro dele.

— Mas eu queria entrar de verdade!

\chapter{O futuro é uma câmera}

— Uma, o que é futuro?

— O futuro é uma câmera.

— Uma câmera?

— É, por onde a gente vê as coisas!

\chapter{Faz tempo que não leio ficção}

Não estou suportando a realidade

\chapter{Que fazer?}

Clima de funeral no país

Na cidade

Na universidade

Ninguém sabe

\chapter{O futuro a quem pertence?}

Imagino Beltrano construindo uma vida para si, podendo se tornar pai de
Uma criança, meu amigo. E deixando todo sofrimento no passado. Uma, seu
pai não está mais longe.

Ou imagino ele morrendo de bala, garrafada, overdose, suicídio. E
deixando todo sofrimento no passado. Uma, seu pai morreu.

\chapter{Sobre as formas do morrer}

— Olha, mamãe, uma minhoca morrida!

— É mesmo, filha, ela tá morta. Vamos?

— Tchau, minhoca, boa mortagem pra você!

\chapter{Desistência}

Hoje pedi para Uma escolher uma boneca ou um bichinho pra dar a uma
bebezinha que a gente ia visitar. Ela pegou o boneco que parece o
Beltrano e deu.

\chapter{Procuro um homem em quem confiar}

Vou escrever no aplicativo

Não

Não vou

\chapter{Logo depois ele morria}

Sonhei que começava uma história com um homem de olhos bons e corpo
aconchegante.

Queria um homem que me amasse

Mas não fosse louco por mim

\chapter{Sexo}

Não faço mais.

Trabalho em casa. Minha vida social acontece no espaço exíguo entre a
porta da escola e a pracinha, onde é bastante improvável encontrar
homens solteiros. Também não me livrei de sensação traumática de que
bastam alguns minutos para o monstro machista aparecer. O que aliás de
fato encontra-se amplamente comprovado pelas minhas próprias
estatísticas. Acresce que vivo sobrecarregada. Fins de semana, 24/\,7;
viroses, 24/\,7; férias e feriados, 24/\,7. Às vezes tiro um dia de descanso
nas duas ou três horinhas que me sobram entre as tarefas de casa e o
trabalho que vou deixar de fazer com uma boa carga de culpa. Não há
repouso. Só de pensar em abrir espaço na minha vida para uma terceira
pessoa me dá um desânimo... Para piorar o quadro, quando fecho os olhos
e penso em interagir com outro corpo, é do corpo dele que sinto falta.
Uma falta que vai diminuindo, mas nunca some completamente. Já tive um
encontro assim uns vinte e cinco anos atrás, com um homem que eu sequer
respeitava. Não some.

\chapter{Ipsis litteris}

\textbf{M.} Anna, você está linda; e mãe! Você não sabe o quanto eu fico
feliz em saber disso. Acho que podemos ser amigos.

Grande abraço.

\textbf{P.} Oi Max, tudo bem?

\textbf{M.} Tudo bem, Anna! Fiquei muito impressionado hoje, pois fui
para o trabalho pensando em você. Em coisas que conversávamos. Que
surpresa a minha, um dos colegas hoje: "Então você conhece a Anna? Sabe
que ela é mãe?" Fiquei paralisado, depois muito feliz. Queria muito ver
você, mãeanna. Sempre me lembro de uma de nossas conversas sobre filhos.

\textbf{P.} Você foi pro trabalho pensando em mim? Por quê?

\textbf{M.} Você é uma das minhas ideias em X.

\textbf{P.} Eu fico meio intrigada. Já faz mais de vinte anos e volta e
meia você aparece dizendo que pensa em mim. Por que isso?

\textbf{M.} Uma ocasião conversamos e eu disse, insisti, que queria que
você me apresentasse a sua família, não sei se você se lembra disso.
Fomos à sua casa.

\textbf{P.} Sim. Você também conheceu a minha mãe.

\textbf{M.} E seu pai, no teatro. Foi um encontro no corredor. Já não
estávamos mais juntos.

\textbf{P.} Mas você não respondeu a minha pergunta.

\textbf{M.} Minha resposta tem relação à sua pergunta é um dos Xs
machadianos da minha cabeça. Os outros rapazes conheciam a sua família,
mas eu não.

\textbf{P.} Talvez porque você tivesse namorada e ficasse me dando um pé
na bunda num dia pra voltar no outro. E mesmo assim você conheceu.

\textbf{M.} Naquele momento, eu não estava com ela, que tinha problemas
em assumir a relação com um negro.

\textbf{P.} Então você casou, formou uma família linda, fez sua vida e
não entendo por que você volta e meia vem me dizer que ainda pensa em
mim. Essa história passou, minhas lembranças não são tão boas quanto as
suas, então deixa passar, já foi.

\textbf{M.} Lembro de você quando leio Machado de Assis, lembro de você
quando alguma aluna diz que quer ser bailarina, mas acha que ``não tem
mais idade", duas no ano passado. Falei-lhes de você.

Lembro de você, por dançarmos, rirmos. E por ser uma pessoa sensível e
inteligente. Quando eu a conheci, havia parado de dançar; voltei e tenho
dificuldade em parar até hoje. O corpo está dando um incentivo para eu
esquecer a ideia de uma vez por todas. Há pouco tempo atrás coreografei
e dancei Otelo. Foi uma pesquisa interessante.

\textbf{P.} Lembro que a gente dançava bem junto. Mas também lembro que
eu era uma menina e você já era um homem feito, que você me botava para
baixo, criticava meu corpo, competia intelectualmente em vez de
estabelecer a parceria que eu propunha e que você disse que queria que
eu fosse uma neguinha

\textbf{M.} Eu me lembro das pessoas em um movimento de que a vida não é
um queijo suíço, cheia de buracos.

\textbf{P.} E sim, lembro que o sexo era fabuloso, mas isso não segura
tudo. Então passou, deixa ir.

\textbf{M.} Interessante é que sempre te achei intelectualmente superior
a mim, inclusive por ter tido uma formação muito melhor. Veja que
interessante, era bom sim. Mas não é pelo sexo que me lembro de você. É
pela dança, pelo seu humor... Eu não queria que você fosse negrinha, nem
poderia querer. Ninguém vira negro, nascemos negros e vamos nos tornando
negros no percurso da vida. Nos tornamos por sermos e pela sociedade que
nos lembra disso. Acho que não há motivos para não sermos amigos, tenho
de você as melhores lembranças. Não me lembro de você pelo sexo, mas sim
pela dança, por sua delicadeza, pela poesia. Você é uma pessoa linda,
alguém com quem vale a conversa. Como toda bailarina, traz seu poço de
angústia, mas, penso que é exatamente esta capacidade de nos
angustiarmos que nos faz melhores, não do que os outros, mas do que nós
mesmos.

Pensei, muito, nas situações nas quais penso em você, nas linhas acima,
nas quais constantemente cito "Brás Cubas". Por que quando a questão são
as memórias e a vida, penso em você? O "X" é uma equação, é bom quando
se resolve. Sempre vi você como alguém em divórcio, e a negação da
maternidade era parte do que eu via deste divórcio com a vida. Tornava-a
um "Capítulo das negativas".

Eu pensava que era pela dança, pelo sorriso; mas não; era pela vida.
Que, junto com as memórias póstumas, traziam-me a lembrança de você, que
diz que "era uma menina"; "O menino é o pai do homem", ensina Machado;
Rosa, em outras palavras, segue pelo mesmo caminho "Passarinho que se
debruça - o voo já está pronto!". As lembranças, nas situações
machadianas - de negação, tem relação com o lado mais triste que conheci
de você. Sim, eu tenho uma família linda. Agora você também tem, o mais
são memórias póstumas.

Seja feliz!

\chapter{Deusa, bençoe}

Uma, minha criança:

Minha mãe cometeu muitos erros comigo e sei que vou cometer muitos
outros com você. Só peço à Nossa Senhora dos descaminhos inconscientes
que sejam erros outros, porque eu não pretendo te negar.

\chapter{Meu sonho erótico}

Estava falando com um amigo sobre um intelectual que a gente admirava
muito e comentava, sem muita importância, que ele havia feito uma
generalização boba ao dizer que mulheres não gostam de água gelada.

Meu amigo tentava ponderar dizendo que, em geral, mulheres sentem mais
frio que homens etc. Eu o olhava divertida, dizia "Você acha mesmo?",
tirava minha roupa e pulava de cabeça num lago muito frio.

Havia outros homens lá e todos achavam espirituosa a minha provocação
enquanto admiravam minha coragem.

Então nadei segura, na certeza de que nada disso seria interpretado como
um convite, e que ficar nua ou demonstrar minha força não me punha em
risco.

Quando saí, um dos homens presentes veio atrás de mim, mas porque ele
tinha ficado encantado com meu senso de humor, com minha força, com meu
desnudamento. O desafio e o despudor tinham sido para ele extremamente
eróticos. A gente saiu conversando e eu caminhava nua.

Não havia a menor possibilidade de assédio no meu sonho. No meu sonho,
minha potência erótica não me tornava alvo de violência.
